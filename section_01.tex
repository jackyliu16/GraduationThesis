\section{引言} % 一级标题,可以用自己的章节标题去替换

\subsection{{\LaTeX}简介}  % 二级标题,可以用自己的章节标题去替换
{\LaTeX}是一种文字排版系统,是基于{\TeX}排版系统并由此发展而来。
与其他文字处理系统相比,{\LaTeX}具有非常明显的优势和弱点,其{\bf 最突出的优势就是高质量、高专业水准的文稿排版效果};
而它{\bf 最大的弱点就是使用可视程度低,致使很多人敬而远之}\cite{bib:1}。
% 引用参考文献时使用\cite{参考文献的标签}

当你无法在Microsoft Word和{\LaTeX}排版系统之间做出选择时,可以利用表\ref{tab:1} % 引用图时使用\ref{表的标签} 
提供的内容进行二者之间的特征对比。

% \begin{table}[ht]  % h和t这两个参数的意思是将表放在当前位置或某页的上方
%     \centering
% 		\caption{Microsoft Word和Latex的主要特征对比}
%     \begin{tabular*}{15cm}{lll}
%     \toprule  
%           特征	&   Microsoft Word	     &  Latex       \\   \midrule  % 表头
%  撰写的文档篇幅 & 迅速看到编辑的结果。   & 通过构建过程才能看到输出。     \\  %第一行
%       易于使用  & 基本功能便于掌握。     & 需要花费一些时间来掌握命令。  \\  % 第二行
% 			排版质量  & 远没有达到专业级别。 &	提供专业级别的排版效果。               \\  % 第三行
% 			科学特性	& 对参考文献的引用支持较弱。& 参考文献由BibTeX管理。\\  % 第四行
% 			价格	& 需要支付费用。	           &  免费使用Latex。\\  % 第五行
% 			兼容性 & 存在兼容性问题。 & 不存在兼容问题。\\ % 最后一行 	
% 		\bottomrule
%     \end{tabular*}
% 		\label{tab:1} % 表的标签,需要在全文中唯一,即不和其他任何标签同名
% \end{table}


% \subsection{{\LaTeX}的缺点}
% {\LaTeX}的缺点体现在两个方面:起点门槛较高和可视性差。
% \begin{itemize}  % 不带序号的列表
% 	\item 起点门槛较高:就算是编写很简单的文章,也要花费较多的时间和精力去学习那么枯燥的命令和使用方法,特别是编辑公式较容易出错。
% 	\item 可视性差:需要经过编译才能看到最终的排版结果。
% \end{itemize}

% \subsection{{\LaTeX}的优点}
%  {\LaTeX}拥有诸多优点,包括排版质量好、提供易用的注释功能、格式自动处理等等。

% \begin{description}  % 用于对若干条目集中进行解释
% 	\item[排版质量高:] {\LaTeX}提供的排版质量表现在对版面尺寸的严格控制,对字距、词距、行距和段距等字符间距松紧适中的掌握,对数学式的精确细致设计等等。
% 	\item[具备注释功能:] 在{\LaTeX}源文件(.tex文件)中,可在任何地方使用注释标记(以``\%''开头),将注释内容完整地保留下来。
% 	\item[格式自动处理:] 作者只需选定文稿的类型,就可以专心内容的创作,至于论文的格式等细节均由{\LaTeX}统一设置。
% 	\item[数学公式精美:] 使用{\LaTeX}排版产生数学公式,其效果精致细腻,而且数学公式越复杂这个特点越明显。
% 	\item[参考文献管理便捷:] {\LaTeX}自带一个辅助工具程序BibTex,它可以为作者创建一个参考文献数据库,作者可以自行填充这个数据库,也可以从网上下载内容进行填充。
% 	\item[通用性强:] 含有各种语言文字的{\LaTeX}源文件可以毫无阻碍地跨系统使用。
% \end{description}

% \subsection{安装MiKTeX和其他工具软件(适用于Windows)}   % 二级标题,可以用你自己的章节标题去替换

% \subsubsection{MiKTek排版系统}
% 访问MiKTeX项目主页中的下载(Download)页面:https://miktex.org/download,下载与你的计算机操作系统相匹配的版本,参见图\ref{fig:1}。%引用图片时使用\ref{图的标签}

% \begin{figure}[hb]   %% h和t这两个参数的意思是将表放在当前位置或某页的下方
%     \centering  % 图片居中放置
%     \includegraphics[scale=0.98]{imgs/MiKTeXDownLoad.jpg}%图片MiKTeXDownLoad.jpg在当前目录下的imgs子目录里面,也可以使用.png格式的图片,scale指定缩放比例 
%     \caption{选择与操作系统相匹配的MiKTeX版本进行下载}  % 图的标题在图的下方
% 		\label{fig:1}  % 图的标签,需要在全文中唯一,即不和其他任何标签同名
% \end{figure}

% 下载完毕后进行安装(参见图\ref{fig:2}),

% \begin{figure}[ht]
%     \centering  % 图片居中放置
%     \includegraphics[scale=0.95]{imgs/Installation.jpg}%图片MiKTeXDownLoad.jpg在当前目录下的imgs子目录里面,也可以使用.png格式的图片 
%     \caption{选择与操作系统相匹配的MiKTeX版本进行下载}  % 图的标题在图的下方
% 		\label{fig:2}  % 图的标签,需要在全文中唯一,即不和其他任何标签同名
% \end{figure}

% 剩余还需要进行的操作包括以下步骤:
% % 带有序号的列表,形如(1)  (2) ...
% % \begin{enumerate}[(1)] 
% % 	\item 看见``设置安装目录''的界面(此处省略),在此处指定将MiKTeX安装在你认为合适的目录下面。
% % 	\item 接下来的步骤\ldots
% % 	\item 最后的步骤\ldots
% % \end{enumerate}

% % 还可以使用带有序号的列表,形如(i)  (ii) ...
% %\begin{enumerate}[i.] 
% %	\item a
% %	\item b	
% % \end{enumerate}

% % 带有序号的列表,形如 a)  b) ...
% %\begin{enumerate}[a)] 
% %	\item a
% %	\item b	
% %\end{enumerate}

% % 带有序号的列表,形如 [1] [2] .
% %\begin{enumerate}  
% %	\renewcommand{\labelenumi}{[\theenumi].} 
% %	\item a
% %	\item b
% %\end{enumerate}


% \subsection{关于引入公式和算法的说明}

% 在论文撰写过程中,有时会需要编辑公式或者引入算法描述,这里我们举两个例子来说明用法。

% \begin{equation}
% SSIM(x,y)=\frac{\left(2\mu_x\mu_y+c1\right)\left(\sigma_{xy}+c2\right)}
% {\left(\mu_x^2+\mu_y^2+c1\right)\left(\sigma_x^2+\sigma_y^2+c2\right)}
% \label{eq:1} % 公式的标签,全文唯一
% \end{equation}

%  % 注意下面这行不能缩进,所以使用了\noindent命令。
% \noindent 其中$\mu_x$代表x的均值, $\mu_y$ 代表y的均值,而$\sigma_x^2$是x的方差\ldots
% 如果需要引用上面的公式,可以这样使用:关于{SSIM(x,y)}的计算,请参见公式\ref{eq:1}。

% 有的时候需要介绍某个算法的设计思想,此时可以参考算法\ref{alg:1}的做法来展示。
% % \begin{algorithm}
% %         \begin{algorithmic}[1] %每行显示行号
% %             \Require $Array$数组,$n$数组大小
% %             \Ensure 逆序数
% %             \Function {MergerSort}{$Array, left, right$}
% %                 \State $result \gets 0$
% %                 \If {$left < right$}
% %                     \State $middle \gets (left + right) / 2$
% %                     \State $result \gets result +$ \Call{MergerSort}{$Array, left, middle$}
% %                     \State $result \gets result +$ \Call{MergerSort}{$Array, middle, right$}
% %                     \State $result \gets result +$ \Call{Merger}{$Array,left,middle,right$}
% %                 \EndIf
% %                 \State \Return{$result$}
% %             \EndFunction                      
% %         \end{algorithmic}
% % 				\caption{归并排序}
% % 				\label{alg:1}
% %     \end{algorithm}


