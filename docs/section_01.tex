\section{绪论}
这个部分直接从开题报告上面抄下来的,还没改

\subsection{项目背景及意义}
TODO 重新引用,检查内容是否合适

今年来,由于一系列计算机网络安全事件的出现,操作系统领域的安全性逐渐被国家重视。
无论是2017年知名闭源操作系统windows中爆出的 永恒之蓝 病毒,还是近年来伊朗核试验基地遭遇“震网”病毒袭击时间,亦或是最近这段时间 linux 操作系统所面临的开源软件 xz 源供应链投毒事件,无不告诉我们操作系统领域安全性的重要性。
作为国民生产生活所不可或缺的一部分,一旦我们没有办法能保障操作系统的安全性,个人计算机尚且可能会被窃取隐私数据,大型的科研计算机亦或是国防科工设备一旦由于操作系统上的后门而遭到入侵,必然会导致巨大的损失。
因此,操作系统作为现代化行业的重要生产工具,其安全性,稳定性和可靠性必然需要得到保障。

而随着 rust 这种新型的,通过强制类型类型声明,所有权检查机制,生命周期检查器等一系列方式保障安全性的编程语言的出台,逐渐有使用此类编程语言进行操作系统设计实现的实验。如 (LANKES 等, 2019) 就在其论文中剃刀了
使用 rust 高级语言代替在 unikernel 中进行开发常用的 c 语言的设想。甚至在后来(LANKES 等, 2020) 尝试将其最初的 HermitCore 重新通过 Rust 语言进行了相关的实现 (Rusty Hermit),结果证明 Rust 语言实现版本与 C 语言实现版本之间
不存在性能差异。同时,由于 Rust 独特的标记不安全代码的设定,在操作系统中有且仅有 3.27% 的代码的安全性需要特殊注意。基于类似于 (BOOS 等,2020) 提出的 Theseus 操作系统,与 (KUENZER 等, 2021) 等提出的 Unikraft 微库操作系统实现理念的理解
,贾越凯博士实现了一种将组件化与库操作系统结合在一起的,基于 Rust 安全性保障的新型操作系统 ArceOS。

这种操作系统类似于较早年间外核(ExoKernel)的设计类似,由于其主要设计在嵌入式平台等专用平台上进行使用,而非基于某种通用的硬件设备或者 VMM 虚拟机的尝试,在实现层面上面临了许多硬件层面的适配问题。

本次毕业设计主要是想要通过在这种不同于传统操作系统架构,如宏内核或者微内核的类 LibOS,Unikernel 架构下操作系统中实现一些简单嵌入式设备常见驱动的过程,为今后的学习打下坚实的基础,也同时为我国现代化操作系统的发展实现添加我的一份力量。

% \subsection{国内外研究现状}

\subsubsection{Unikernel 国内外研究现状分析}

(1)(ANDERSON 等)(MASSALIN 等, 1989) 认为,通用操作系统中的抽象实现迫使不需要特定功能的应用程序支付大量的开销
(2)(ENGLER 等) 通过一种 Secure Bindings 的机制对于安全进行保障,通过这个机制,使得LibOS能够与其他资源捆绑在一起,实现原先由内核保障的安全性(to separate protection from managerment),同时释放了应用程序的性能,灵活性以及功能[ 使之不再受到操作系统层面高级别抽象的局限而可以细粒度的操作],赋予了应用程序更多的控制权。但是这样的内核由于额外产生的上下文切换导致应用程序之间的通信变得极为复杂[ PESSÉ S, XIA Y, QIU L, 2015. 计算机系统原理讲义[EB/OL]//计算机系统原理(课程讲义). (2015-07-21)[2023-12-06]. https://unitial.gitbooks.io/csp/content/index.html.]。
(3)(MADHAVAPEDDY 等) 提出了Unikernel这种专用的,单地址空间的,使用Library OS 构建的镜像,提供了一个小型,安全,快速的工作负载。
但是与之前ExoKernel等传统库操作系统不同的是,这种工作负载被指定工作在标准虚拟机管理程序上,以避免可能遇到的硬件兼容问题,
从而实现安全[ 针对于单一应用进行服务,不提供类似于Shell之类的冗余功能,避免了代码注入攻击,实现了更大程度上的安全保证。],紧凑[ 相较于一般的BIND DNS的Linux VM 镜像达到462MB,根据他们方法生成的Mirage设备生成的镜像大小只要183.5kB(没有实现所有功能集,但是包含了queryperf测试套件的必要功能,且可用于自托管项目)]的高度专业化单一用途设备虚拟机。
(4)(PORTER 等) 在他们的研究中通过将广泛使用的单片操作系统(Windows 7)重构为一个功能丰富的库操作系统(Library OS),
由一个小型抽象集(线程,虚存,I/O流)连接 Library OS 和 Host OS实现了独立于底层内核组件。
在早期的 Library OS 支持者中认为,通过个性化定制每一个应用程序,可以实现OS的较好性能,
但是现在Library OS已经成为了现代虚拟机监视器的牺牲品。 作者在本文中通过将每个Library OS的特性(应用程序依赖的) 共享底层的 Host OS 资源,这样在操作中,只需要根据应用程序的API进行提取,这样相较于虚拟化整个OS会大大的降低开销。
% \footnote{\href{source}{https://blog.csdn.net/qq_40119224/article/details/118754160}}
(5)(KUENZER 等, 2021) 在研究中提出了一种被称为Unikraft的新型微库操作系统,这种操作系统提供了完全模块化的操作系统基元,因此很容易对于内核进行定制,同时还暴露了一组可组合,面向性能的API,方便开发者提高性能。
(6)(BOOS 等,2020) 提出了一种名为Theseus的操作系统,他通过减少一个组件为另一个组件保留的状态来重新设计和改进操作系统模块化。在这篇文章中作者通过设计将尽可能多的操作系统职能转嫁给编译器,实现了对于核心操作系统组件的实时演化和故障恢复。
(7)(宋, 2017)在论文中提到,一般用于点名的一卡通系统本身存在较多缺陷,比如说卡丢失,代刷卡等。在科学技术发达的今天,学校的考勤管理工作基本上仍处于人工时代或者半人工时代,相较于社会上的企事业单位已经采用了基于摄像头和人工智能的智能工时记录系统或者基于虹膜识别的考勤打卡系统而言具有滞后性。

\subsection{研究目标}

1、研究的主要内容
本次研究主要分为以下几个目标,搭建嵌入式环境,编写指纹识别驱动以及网卡驱动实现。

2、研究的预期目标
能够在同一网段下实现或者更小的嵌入式开发板的相互通信,并且能够在Client端获取输入的指纹信息,将其加密后传输到Servers段进行解密,在解密之后Servers段传送解密成功信息给Client端,Client端通过显示屏显示对应信息或者通过指示灯表示打卡成功的信息。

3、研究的创新点
    尝试通过一种新颖的组件化操作系统对于一个简单应用场景进行重构。

\subsection{研究方法}

首先先简单完成电脑中有关于Rust等内容的开发环境配置,同时根据Nix生成一套可重构的开发环境框架,确保在后面的开发过程中整体开发过程是可复现的。
其次,深入了解现有的网卡驱动开发经验,并且了解其中所遇到的问题以及可能性的解决方案。在此基础上对于As608指纹识别模块驱动的现有实现进行了解,了解其中主要使用的数据结构等内容,以方便后面通过rust语言对于现有驱动进行重构。
然后,通过对于现有的指纹识别系统的分析,了解到目前常用的指纹识别框架。尝试了解no\_std情况下以及std情况下开发的差异。同时分析可能需要用到的库大概有多少是需要重新实现或者替换库的。
通过ArceOS现有的工作完成网络联网测试,实现几个简单的用户程序,首先确保能在多台华山派CV1811H甚至支持更少功能的嵌入式设备上运行简单发包程序并且联网成功。根据已有驱动尝试在ArceOS上进行测试等开发工作,最终将指纹识别驱动以及可能会用到的显示屏驱动,装载上进行整机测试。

引用图\ref{fig},表\ref{tab},公式\eqref{eq:0}
