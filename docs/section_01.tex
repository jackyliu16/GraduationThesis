\section{绪论}
这个部分直接从开题报告上面抄下来的,还没改

\subsection{项目背景及意义}
TODO 重新引用,检查内容是否合适

当前,随着经济的飞速发展,我国在操作系统方面的研究问题日益突出,成为一个制约国家经济发展的“瓶颈”。作为工业生产和任命生活不可或缺的基础设施更是如此。

近年来,随着用户逐渐由于购买硬件占用空间或者不符合成本效益,用户不想在本地管理服务器,或者想要通过分布式方法快速部署服务器等因素逐渐转向现有的云端业务厂商。像这种云服务厂商可以通过他们所具有的大量运算资源和虚拟化技术充分利用其硬件资源,以相较于一般公司而言更低的成本提供服务。但是,就大部分使用情况而言, 使用者租借服务器只是为了假设单一的网站或者服务,即在他们的hypervisor中往往只有一个应用程序在运行。在这种情况下为了单一应用服务部署整个Linux Kernel的行为并没有实现极致的性能,因为 Linux Kernel 实际上是一种通用计算机操作系统内核,旨在服务于各种各样的运算设备,一般程序并不需要使用到Linux Kernel中的所有功能[ 比如说提供TCP echo Server的应用程序不需要理解不同的文件系统架构],同时,Linux Kernel是一种实现了用户 - 内核隔离的操作系统,但是在单一应用程序情况下,其实通过用户 - 内核隔离实现安全性保障是没有必要的,因为唯一在操作系统中运行的程序是由部署方提供的,通过隔离保证自己提供的操作系统的安全性是没有效率的。

这种操作系统拒绝了操作系统对于特定应用进行优化的优势,通过要求应用程序根据现有抽象的实现进行修改,限制了应用进程构建器的灵活性,导致新的抽象只能通过在现有抽象之上的笨拙仿真来进行添加(ENGLER 等)。因此,(PORTER 等) 尝试通过将Windows7重构成为一个功能丰富的操作系统以实现更好的基于应用程序的定制效果。同时(MADHAVAPEDDY 等)也提出了一种基于特定语言的,能在云端快速部署的Unikernel,这种工作负载可以在具有更高安全性和紧凑程度下完成同等与原始LinuxKernel的功能。之后,(KUENZER 等, 2021) 的研究中提出了一种Unikraft的微库操作系统,这种操作系统通过链接的方式将完全模块化的操作系统基元组合在了一起,方便了用户对于操作系统进行定制,但是由于其主要使用c语言进行完成,内部存在大量的条件编译,对于用户的使用以及维护带来了影响。

近来,(LANKES 等, 2019)在论文中提出了使用Rust高级语言替代在unikernel开发中常用的c语言的设想之后,(LANKES 等, 2020)尝试提供了原有HermitCore的一个Rust语言改写(Rusty Hermit)。证明了Rust语言实现版本和C语言实现版本之间不存在性能差异的构建常见Rust应用程序。在他们的实现中,基于Rust显示标记不安全代码的规定,在内核中仅存在3.27%的区域需要密切关注安全性。基于类       似的实现目的,贾越凯[ 清华大学陈瑜老师的博士生,目前论文尚未发表]提供了一种将组件化和库操作系统集合集合在一起,并且通过Rust提供安全性保证的操作系统ArceOS。这种操作系统某种程度上类似于(BOOS 等,2020)的细粒度实现。
针对于ArceOS这种操作系统在车用嵌入式设备上的尝试虽然已经有了些许的尝试,但是可能由于类似于ExoKernel所面临的困境一样,ArceOS没有使用Unikernel类似的基于VMM的尝试,同样面临了许多硬件层面的适配问题。

因此我想通过本次毕业设计的学习,深入了解有关于其中的权衡以及对于硬件层面的适配,为今后的学习打下夯实的基础。同时也为我国现代操作系统的实现增添宝贵的一份力量。

\subsection{国内外研究现状}

(1)(ANDERSON 等)(MASSALIN 等, 1989) 认为,通用操作系统中的抽象实现迫使不需要特定功能的应用程序支付大量的开销
(2)(ENGLER 等) 通过一种 Secure Bindings 的机制对于安全进行保障,通过这个机制,使得LibOS能够与其他资源捆绑在一起,实现原先由内核保障的安全性(to separate protection from managerment),同时释放了应用程序的性能,灵活性以及功能[ 使之不再受到操作系统层面高级别抽象的局限而可以细粒度的操作],赋予了应用程序更多的控制权。但是这样的内核由于额外产生的上下文切换导致应用程序之间的通信变得极为复杂[ PESSÉ S, XIA Y, QIU L, 2015. 计算机系统原理讲义[EB/OL]//计算机系统原理(课程讲义). (2015-07-21)[2023-12-06]. https://unitial.gitbooks.io/csp/content/index.html.]。
(3)(MADHAVAPEDDY 等) 提出了Unikernel这种专用的,单地址空间的,使用Library OS 构建的镜像,提供了一个小型,安全,快速的工作负载。
但是与之前ExoKernel等传统库操作系统不同的是,这种工作负载被指定工作在标准虚拟机管理程序上,以避免可能遇到的硬件兼容问题,
从而实现安全[ 针对于单一应用进行服务,不提供类似于Shell之类的冗余功能,避免了代码注入攻击,实现了更大程度上的安全保证。],紧凑[ 相较于一般的BIND DNS的Linux VM 镜像达到462MB,根据他们方法生成的Mirage设备生成的镜像大小只要183.5kB(没有实现所有功能集,但是包含了queryperf测试套件的必要功能,且可用于自托管项目)]的高度专业化单一用途设备虚拟机。
(4)(PORTER 等) 在他们的研究中通过将广泛使用的单片操作系统(Windows 7)重构为一个功能丰富的库操作系统(Library OS),
由一个小型抽象集(线程,虚存,I/O流)连接 Library OS 和 Host OS实现了独立于底层内核组件。
在早期的 Library OS 支持者中认为,通过个性化定制每一个应用程序,可以实现OS的较好性能,
但是现在Library OS已经成为了现代虚拟机监视器的牺牲品。 作者在本文中通过将每个Library OS的特性(应用程序依赖的) 共享底层的 Host OS 资源,这样在操作中,只需要根据应用程序的API进行提取,这样相较于虚拟化整个OS会大大的降低开销。
% \footnote{\href{source}{https://blog.csdn.net/qq_40119224/article/details/118754160}}
(5)(KUENZER 等, 2021) 在研究中提出了一种被称为Unikraft的新型微库操作系统,这种操作系统提供了完全模块化的操作系统基元,因此很容易对于内核进行定制,同时还暴露了一组可组合,面向性能的API,方便开发者提高性能。
(6)(BOOS 等,2020) 提出了一种名为Theseus的操作系统,他通过减少一个组件为另一个组件保留的状态来重新设计和改进操作系统模块化。在这篇文章中作者通过设计将尽可能多的操作系统职能转嫁给编译器,实现了对于核心操作系统组件的实时演化和故障恢复。
(7)(宋, 2017)在论文中提到,一般用于点名的一卡通系统本身存在较多缺陷,比如说卡丢失,代刷卡等。在科学技术发达的今天,学校的考勤管理工作基本上仍处于人工时代或者半人工时代,相较于社会上的企事业单位已经采用了基于摄像头和人工智能的智能工时记录系统或者基于虹膜识别的考勤打卡系统而言具有滞后性。

\subsection{研究目标}
引用图\ref{fig},表\ref{tab},公式\eqref{eq:0}

1、研究的主要内容
本研究主要分为以下几个部分,嵌入式系统搭建,PC机程序设置,编写指纹识别模块驱动以及网卡驱动。

2、研究的预期目标
能够在同一网段下实现或者更小的嵌入式开发板的相互通信,并且能够在Client端获取输入的指纹信息,将其加密后传输到Servers段进行解密,在解密之后Servers段传送解密成功信息给Client端,Client端通过显示屏显示对应信息或者通过指示灯表示打卡成功的信息。

3、研究的创新点
    尝试通过一种新颖的组件化操作系统对于一个简单应用场景进行重构。

\subsection{研究方法}

首先先简单完成电脑中有关于Rust等内容的开发环境配置,同时根据Nix生成一套可重构的开发环境框架,确保在后面的开发过程中整体开发过程是可复现的。
其次,深入了解现有的网卡驱动开发经验,并且了解其中所遇到的问题以及可能性的解决方案。在此基础上对于As608指纹识别模块驱动的现有实现进行了解,了解其中主要使用的数据结构等内容,以方便后面通过rust语言对于现有驱动进行重构。
然后,通过对于现有的指纹识别系统的分析,了解到目前常用的指纹识别框架。尝试了解no\_std情况下以及std情况下开发的差异。同时分析可能需要用到的库大概有多少是需要重新实现或者替换库的。
通过ArceOS现有的工作完成网络联网测试,实现几个简单的用户程序,首先确保能在多台华山派CV1811H甚至支持更少功能的嵌入式设备上运行简单发包程序并且联网成功。根据已有驱动尝试在ArceOS上进行测试等开发工作,最终将指纹识别驱动以及可能会用到的显示屏驱动,装载上进行整机测试。

引用图\ref{fig},表\ref{tab},公式\eqref{eq:0}