\nonumsection{附录 - 部分代码}

\begin{lstlisting}[language=nix
    , caption={可重建编译环境配置文件}
    , label = {nix-flake}
    , numbers = left
    , breaklines=true
    , breakatwhitespace=true]
{
description = "ArceOS Development Environment";

inputs = {
  nixpkgs.url      = "github:NixOS/nixpkgs/nixos-unstable";
  nixpkgs-qemu7.url = "https://github.com/NixOS/nixpkgs/archive/7cf5ccf1cdb2ba5f0
  8f0ac29fc3d04b0b59a07e4.tar.gz";
  rust-overlay.url = "github:oxalica/rust-overlay";
  flake-utils.url  = "github:numtide/flake-utils";
};

outputs = { self, nixpkgs, nixpkgs-qemu7, rust-overlay, flake-utils, ... }:
  flake-utils.lib.eachDefaultSystem (system:
  let
    overlays = [ 
    (import rust-overlay)
    (self: super: {
      # ref: https://github.com/the-nix-way/dev-templates
      rust-toolchain =
      let
          rust = super.rust-bin;
      in
      if builtins.pathExists ./rust-toolchain.toml then
        rust.fromRustupToolchainFile ./rust-toolchain.toml
      else if builtins.pathExists ./rust-toolchain then
        rust.fromRustupToolchainFile ./rust-toolchain
      else
        # The rust-toolchain when i make this file, which maybe change
        (rust.nightly.latest.override {
          extensions = [ "rust-src" "llvm-tools-preview" "rustfmt" "clippy" ];
          targets = [ "x86_64-unknown-none" "riscv64gc-unknown-none-elf" "aarch64-unknown-none-softfloat" ];
        });
      qemu7 = self.callPackage "${nixpkgs-qemu7}/pkgs/applications/virtualization/qemu" {
        inherit (self.darwin.apple_sdk.frameworks) CoreServices Cocoa Hypervisor;
        inherit (self.darwin.stubs) rez setfile;
        inherit (self.darwin) sigtool;
        # Reduces the number of qemu source files from ~10000 to ~3619 source files.
        hostCpuTargets = ["riscv64-softmmu" "riscv32-softmmu" "x86_64-softmmu" "aarch64-softmmu" ];
      };
      x86_64-linux-musl-cross = fetchTarball {
        url = "https://musl.cc/x86_64-linux-musl-cross.tgz";
        sha256 = "172zrq1y4pbb2rpcw3swkvmi95bsqq1z6hfqvkyd9wrzv6rwm9jw";
      };
      aarch64-linux-musl-cross = fetchTarball {
        url = "https://musl.cc/aarch64-linux-musl-cross.tgz";
        sha256 = "05cwryhr88sjmwykha5xvfy4vcrvwaz92r9an7n5bsyzlwwk0wpn";
      };
      riscv64-linux-musl-cross = fetchTarball {
        url = "https://musl.cc/riscv64-linux-musl-cross.tgz";
        sha256 = "119y1y3jwpa52jym3mxr9c2by5wjb4pr6afzvkq7s0dp75m5lzvb";
      };
    })
    ];
    pkgs = import nixpkgs {
      inherit system overlays;
    };
  in
{
devShells.default = pkgs.mkShell {
  buildInputs = (with pkgs;[
    # Basic
    openssl pkg-config fd zlib gnumake
    # Development tools
    ripgrep fzf zellij
    # Rust
    rustup
    cargo-binutils
    rust-toolchain
    # Test
    apacheHttpd
  ]) ++ [
  # Overlays part
    pkgs.qemu
  ];
  
  shellHook = ''
    alias find=fd
    export SHELL=zsh

    # Change the mirror of rust
    export RUSTUP_DIST_SERVER=https://mirrors.ustc.edu.cn/rust-static
    export RUSTUP_UPDATE_ROOT=https://mirrors.ustc.edu.cn/rust-static/rustup

    unset OBJCOPY # Avoiding Overlay
    export LIBCLANG_PATH="${pkgs.llvmPackages.libclang.lib}/lib" # nixpkgs@52447
    export LD_LIBRARY_PATH="${pkgs.zlib}/lib:$LD_LIBRARY_PATH" # nixpkgs@92946
  
    export PATH=$PATH:${pkgs.aarch64-linux-musl-cross}/bin:
    ${pkgs.riscv64-linux-musl-cross}/bin:${pkgs.x86_64-linux-musl-cross}/bin
  '';
};
};
  );
}      
\end{lstlisting}

\begin{lstlisting}[language=C
        , caption={初始化 RX Rings}
        , label={code::InitRxRings}
        , numbers = left
        , breaklines=true
        , captionpos=b
        , breakatwhitespace=true]
static int bcmgenet_init_rx_ring(struct bcmgenet_priv *priv, unsigned int index, unsigned int size, unsigned int start_ptr, unsigned int end_ptr);
for (i = 0; i < priv->hw_params->rx_queues; i++) { // 初始化函数
ret = bcmgenet_init_rx_ring(priv, i, priv->hw_params->rx_bds_per_q,
              i * priv->hw_params->rx_bds_per_q, (i + 1) *
              priv->hw_params->rx_bds_per_q);
if (ret) return ret;
ring_cfg |= (1 << i);
dma_ctrl |= (1 << (i + DMA_RING_BUF_EN_SHIFT));
\end{lstlisting}

\begin{lstlisting}[language=Rust
  , caption={tock register 包装}
  , label={code::tock_register}
  ]
  register_structs! {
    Channel {
        (0x00 => CS: ReadWrite<u32, CS::Register>),
        (0x04 => CONBLK: ReadWrite<u32, CONBLK::Register>),
        (0x08 => TI: ReadWrite<u32, TI::Register>),
        (0x0c => S_AD: ReadWrite<u32, S_AD::Register>),
        (0x10 => D_AD: ReadWrite<u32, D_AD::Register>),
        ...
      },
  } ...
  register_bitfields! { u32,
  CS [ // Control and Status registers
    RESET OFFSET(31) NUMBITS(1),
    ABORT OFFSET(30) NUMBITS(1),
    DISDEBUG OFFSET(29) NUMBITS(1),
    WAIT_FOR_OUTS_TANDING_WRITES OFFSET(28) NUMBITS(1),
    PANIC_PRIORITY OFFSET(20) NUMBITS(3),
    PRIORITY OFFSET(16) NUMBITS(4),
    ...
  ], CS_DMA4 [ ... ] ...}
\end{lstlisting}

\begin{lstlisting}[language=C
  , style=CStyle
  , label={code::eth-send}
  , caption=Ethernet Send Packet SpeedTest]
// include ...
int send_test(double duration, int frame_len) {
    assert(frame_len >= ETH_ZLEN);
    assert(frame_len <= ETH_FRAME_LEN);

    int sockfd;
    struct ifreq if_idx;
    struct sockaddr_ll socket_address;
    // char *interface = "eth0";
    // char *interface = "enp0s20f0u2c2";
    // char *interface = "enp1s0";
    unsigned char src_mac[6] = {0xd8, 0x3a, 0xdd, 0x7f, 0xdd, 0x38};
    unsigned char dest_mac[6] = {0xFF, 0xFF, 0xFF, 0xFF, 0xFF, 0xFF}; // Broadcast address
    unsigned char ether_frame[ETH_FRAME_LEN];
    time_t start_time, current_time;
    unsigned long long frame_count = 0;
    unsigned long long total_bytes = 0;

    printf("\nRunning Ethernet Send Packet SpeedTest, waiting for %.2f for complete the test.\n", duration);

    // 创建原始套接字
    if ((sockfd = socket(AF_PACKET, SOCK_RAW, htons(ETH_P_ALL))) < 0) {
        perror("socket");
        exit(1);
    }

    // 获取网络接口索引
    memset(&if_idx, 0, sizeof(struct ifreq));
    strncpy(if_idx.ifr_name, interface, IFNAMSIZ-1);
    if (ioctl(sockfd, SIOCGIFINDEX, &if_idx) < 0) {
        perror("SIOCGIFINDEX");
        exit(1);
    }

    // 构建目的地址结构体
    memset(&socket_address, 0, sizeof(struct sockaddr_ll));
    socket_address.sll_ifindex = if_idx.ifr_ifindex;
    socket_address.sll_halen = ETH_ALEN;
    memcpy(socket_address.sll_addr, dest_mac, 6);

    // 构建以太网帧
    memcpy(ether_frame, dest_mac, 6);
    memcpy(ether_frame + 6, src_mac, 6);
    ether_frame[12] = 0x08; // Type: 0x0800 IP
    ether_frame[13] = 0x01;

    start_time = time(NULL);
    do {
        // init counter 
        char convertBuffer [64] = {0};
        snprintf(convertBuffer, sizeof(convertBuffer), "%llu", frame_count);
        memcpy(ether_frame + 16, convertBuffer, 64);
        // send ethernet packet
        if (sendto(sockfd, ether_frame, frame_len, 0, (struct sockaddr*)&socket_address, sizeof(struct sockaddr_ll)) < 0) {
            perror("sendto");
            exit(1);
        } else { total_bytes += frame_len; }

        frame_count++;
        current_time = time(NULL);
    } while (difftime(current_time, start_time) < duration);

    printf("Ethernet frame sent successfully\n");

    // double seconds = current_time - start_time + (current_time - start_time) / 1000000.0;
    double rate_mbps = (total_bytes * 8) / (duration * 1000000.0); // total_bytes * 8 / 10 * 1000_000

    printf("Sent %lld frames in %.2f seconds\n", frame_count, duration);
    printf("Sent %llu bytes in total, rate: %.2f MBps\n", total_bytes, rate_mbps);

    close(sockfd);
    return 0;
}

int main() {
    send_test(10.0, ETH_ZLEN);
}
\end{lstlisting}

\begin{lstlisting}[language=C
  , style=CStyle
  , label={code::eth-recv}
  , caption=Ethernet Recv Packet]
// include ...
int recv_loop(double duration) {
    int sockfd;
    char buffer[65535];
    struct sockaddr_ll addr;
    socklen_t addr_len = sizeof(struct sockaddr_ll);
    int capture_started = 0;
    time_t start_time, current_time;
    unsigned long long frame_count = 0;
    unsigned long long loss_count = 0;
    unsigned long long total_bytes = 0;

    // 指定要监听的网络接口
    // const char *interface = "enp0s20f0u2c2";
    // const char *interface = "enp1s0";

    // 创建原始套接字
    if ((sockfd = socket(AF_PACKET, SOCK_RAW, htons(ETH_P_ALL))) == -1) {
        perror("socket");
        return 1;
    }

    // 绑定到指定的接口
    struct sockaddr_ll sa;
    memset(&sa, 0, sizeof(sa));
    sa.sll_family = AF_PACKET;
    sa.sll_ifindex = if_nametoindex(interface);
    sa.sll_protocol = htons(ETH_P_ALL);
    if (bind(sockfd, (struct sockaddr*)&sa, sizeof(sa)) == -1) {
        perror("bind");
        close(sockfd);
        return 1;
    }

    printf("绑定成功\n");

    // 开始捕获数据包
    while (1) {
        // 接收以太网帧
        ssize_t num_bytes = recvfrom(sockfd, buffer, sizeof(buffer), 0, (struct sockaddr*)&addr, &addr_len);
        if (num_bytes == -1) {
            perror("recvfrom");
            break;
        }

        struct ethhdr *eth = (struct ethhdr *) buffer;

        // 在接收到第一个以太网帧后开始计时
        if (!capture_started && ntohs(eth->h_proto) == 0x0801) {
            start_time = time(NULL);
	    printf("start\n");
            capture_started = 1;
        }

        // 检查计时器,10 秒后停止
        current_time = time(NULL);
        if (difftime(current_time, start_time) > duration) {
            break;
        }

        if (ntohs(eth->h_proto) == 0x0801) {
            // handle cnt
            char ConvertBuffer[16] = {0};
            memcpy(&ConvertBuffer, &buffer[16], 8);
            unsigned long long cnt = 0;
            sscanf(ConvertBuffer, "%llu", &cnt);

//printf("except: %llu, get: %llu\n", frame_count, cnt);
            if (cnt != frame_count) {
		unsigned long long diff = 0 ;
		if (frame_count > cnt) {
			diff = frame_count - cnt;
		} else {
			diff = cnt - frame_count;
		}
                printf("Loss %llu:%llu-%llu package\n", diff, frame_count, cnt);
                loss_count += diff;
                frame_count = cnt;
            }

            frame_count++;
            total_bytes += num_bytes;
        }
    }

    printf("Received %lld frames (%lld bytes) and loss %llu package in 10 seconds.\n", frame_count, total_bytes, loss_count);
    double rate_mbps = (total_bytes * 8) / (duration * 1000000.0); // total_bytes * 8 / 10 * 1000_000
    printf("mbps: %f\n", rate_mbps);
    close(sockfd);
    return 0;
}

int main() {
    recv_loop(10);
}
\end{lstlisting}