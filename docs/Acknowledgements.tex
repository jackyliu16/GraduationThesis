\nonumsection{致谢}

写完了这整篇毕业设计,回首而望,不由得感叹,我竟然真的历经重重困难,从某种意义上实现了原初的设计目标。
说实话,在写这篇论文的过程中,我无数次的想要放弃现有的设计,全面转向一个相对来说更为简单的, “大众化”的实现,
十分感谢自己,能够在这样的情况下坚持下来。
虽然在时间等因素的影响下,我没能将我实现的系统置于企业使用的层面上予以实现,所实现的考勤系统在鲁棒性,安全性等
方案上考量不足,但是最终还是实现了一个可在特定情况下使用的,脆弱的指纹考勤系统。

在此,我要向毕设参考,使用的开源作品致以感谢,他们分别是 
\href{https://github.com/rcore-os/arceos.git}{arceos}, 
\href{https://github.com/jonlamb-gh/rpi4-rust-workspace.git}{rpi4-rust-workspace},
\href{https://github.com/raspberrypi/linux.git}{linux-rpi4},
\href{https://github.com/rust-embedded/awesome-embedded-rust.git}{awesome-embedded-rust},
\href{https://github.com/rust-embedded/embedded-hal.git}{embedded-hal},
\href{https://github.com/rust-embedded-community/embedded-nal}{embedded-nal},
\href{https://github.com/japaric/enc28j60.git}{enc28j60},
感谢他们公开分享的源代码,没有了他们,我就不可能那么容易的完成我的毕设。

接着,我要向在毕设的过程中关心过我,帮助过我的人致以由衷的感谢。
感谢\hide{清华大学朱懿}(研一),\hide{广州大学张轩铬同学}(大三),\hide{江西理工大学胡友豪}同学(大三),泉城实验室石磊老师在操作系统开发上
给予我的诸多帮助及本校\hide{生物光子学研究院董卓睿}学长(研二)在硬件模块选型上提出的建议,还感谢\hide{刘川豪}同学在 \LaTeX 上提供的帮助和\hide{刘诺}同学临时借用的 SD 卡。
特别感谢\hide{阿图教育的李明老师}无偿提供的树莓派4B开发套件,让我得以在这块开发板上实现我的整个考勤系统,
感谢我的毕设指导老师\hide{刘刚老师}及\hide{稀谷碳源新材料}有限公司提供的测试场地,使我能高效的进行系统优化,而不需要反复的搬运我的设备。
最后,感谢本校咨询中心的\hide{潘绮敏老师}以及同系的\hide{梁迅}、\hide{朱道鑫}、\hide{彭桐}等同学,
他们在毕设期间的陪伴与支持,让我克服各种挑战,顺利完成毕业设计

也要特别感谢同系的\hide{王天诚},\hide{吴佳芮},\hide{李钰},\hide{黄佳欣},\hide{梁讯},\hide{朱道鑫},\hide{彭桐}等同学和
微信小群中的\hide{罗秋杏},\hide{吕朋林},\hide{董卓睿},\hide{孙柳},\hide{李盼},\hide{吴海军},\hide{陈柳铮},\hide{洪嘉一}等同学。
在这段学术旅程中,他们不仅是我的同窗好友,在我略显愚钝时给我信赖与依靠,
而且每当遇到困难和挑战时,他们总是不厌其烦地倾听我的抱怨,给予建设性的意见和鼓励。
我们一起度过了无数个激动人心的学习时刻,分享着彼此的喜怒哀乐。
他们的陪伴让这段旅程更加难忘,谢谢他们为我带来的友谊和支持。希望我们的友谊能够长存,共同迎接未来的挑战和成就。

除此之外,我还要向在我完成本科学习过程中所受到老师,机构的教导表示感谢。
感谢\hide{汪红松,梁艳,周成菊,杨欢,刘刚}\footnote{软件学院时任教师},\hide{罗胜舟}\footnote{前软件学院教师},
\hide{平功波,周丽婷,卢允之,叶绮文}\footnote{经管学院时任教师}等老师在我四年本科学习中,
或是在在本科教学中十分认真,亦或是其浓郁的个人魅力打动了我,使我坚定了我的道路。
感谢\hide{陈渝,向勇,李明老师等提供的开源操作系统夏令营},我对操作系统很大一部分的理解来源于此和
与之相关的一系列维护者,如\hide{石磊,萧络元,朱懿,闭浩扬}等。
还感谢我一路走来陪伴我的一个个开源项目,如 xv6,CS自学指南,CS61B等,是他们的教导有了我的今天。

最后,感谢岁月,于世间之欢愉与悲苦,许以宽宏。

% 我与操作系统结缘于 2022 年的操作系统开源夏令营,在参加该夏令营的过程中
% 我通过实际在操作系统中完成一些小实验,对于原先干巴巴的操作系统理念有了更多
% 的了解,但是,当年由于我基础相对较为薄弱,外加转专业后补充知识同样需要许多
% 时间,在那之后就没有更近一步的继续了\footnote{在当年还参与了一些小项目,但是由于学校排课因素,没有办法有一个比较多的时间来完成对应学习},后来在
