\nonumsection{致谢}

写完了这整篇毕业设计,回首而望,不由得感叹,我竟然真的历经重重困难,从某种意义上实现了原初的设计目标。
说实话,在写这篇论文的过程中,我无数次的想要放弃现有的设计,全面转向一个相对来说更为简单的, “大众化”的实现,
十分感谢自己,能够在这样的情况下坚持下来。
虽然在时间等因素的影响下,我没能将我实现的系统置于企业使用的层面上予以实现,所实现的考勤系统在鲁棒性,安全性等
方案上考量不足,但是最终还是实现了一个可在特定情况下使用的,脆弱的指纹考勤系统。

在此,我要向在毕设的过程中关心过我,帮助过我的人致以由衷的感谢。
感谢清华大学朱懿(研一),广州大学张轩铬同学(大三),江西理工大学胡友豪同学(大三),泉城实验室石磊老师在操作系统开发上
给予我的诸多帮助及本校生物光子学研究院董卓睿学长(研二)在硬件模块选型上提出的建议。
特别感谢阿图教育的李明老师无偿提供的树莓派4B开发套件,让我得以在这块开发板上实现我的整个考勤系统,
感谢我的毕设指导老师刘刚老师及稀谷碳源新材料有限公司提供的测试场地,使我能高效的进行系统优化,而不需要反复的搬运我的设备。
最后,感谢本校咨询中心的潘绮敏老师以及同系的梁迅、朱道鑫、彭桐等同学,
他们在毕设期间的陪伴与支持,让我克服各种挑战,顺利完成毕业设计。

在此也要特别感谢同系的王天诚,吴佳芮,李钰,黄佳欣,朱道鑫,梁讯,彭桐等同学和
微信小群中的罗秋杏,吕朋林,董卓睿,孙柳,李盼,胡臻,吴海军,陈柳铮,谭沛轩,洪嘉一等同学。
在这段学术旅程中,他们不仅是我的同窗好友,在我略显愚钝时给我信赖与依靠,
而且每当遇到困难和挑战时,他们总是不厌其烦地倾听我的抱怨,给予建设性的意见和鼓励。
我们一起度过了无数个激动人心的学习时刻,分享着彼此的喜怒哀乐。
他们的陪伴让这段旅程更加难忘,谢谢他们为我带来的友谊和支持。希望我们的友谊能够长存,共同迎接未来的挑战和成就。

除此之外,我还要向在我完成本科学习过程中所受到老师,机构的教导表示感谢。
感谢汪红松,梁艳,周成菊,杨欢,刘刚\footnote{软件学院时任教师},罗胜舟\footnote{前软件学院教师},
平功波,周丽婷,卢允之,叶绮文\footnote{经管学院时任教师}等老师在我四年本科学习中,
或是在在本科教学中十分认真,亦或是其浓郁的个人魅力打动了我,使我坚定了我的道路。
感谢陈渝,向勇,李明老师等提供的开源操作系统夏令营,我对操作系统很大一部分的理解来源于此和
与之相关的一系列维护者,如石磊,萧络元,朱懿,闭浩扬等。
还感谢我一路走来陪伴我的一个个开源项目,如 xv6,CS自学指南,CS61B等,是他们的教导有了我的今天。

% 我与操作系统结缘于 2022 年的操作系统开源夏令营,在参加该夏令营的过程中
% 我通过实际在操作系统中完成一些小实验,对于原先干巴巴的操作系统理念有了更多
% 的了解,但是,当年由于我基础相对较为薄弱,外加转专业后补充知识同样需要许多
% 时间,在那之后就没有更近一步的继续了\footnote{在当年还参与了一些小项目,但是由于学校排课因素,没有办法有一个比较多的时间来完成对应学习},后来在
