\section{结论}

本论文提供的成果作为 ArceOS 组件化操作系统驱动程序,在该平台建设中占有一定的地位。
本论文以几种常见嵌入式设备设备驱动为中心,首先构建了编译、调试 ArceOS 组件化操作系统的软件环境
,如交叉编译工具链的构建,Rust程序语言工具链的配置等。
重点分析了如 uboot, linux(rpi), circle 等操作系统在树莓派4B 上应用的以太网驱动与串口传输驱动,
了解了有关驱动程序的关键技术,如mdio,DMA等。
深入分析了网卡设备驱动程序的数据结构、驱动框架等,并基于 Rust 语言,改写了现有实现的以太网驱动程序。

虽然本文的工作取得了初步的成功,完成以太网设备驱动的开发及相关软件编译,调试环境的构建,
但要使 ArceOS 组件化操作系统的功能更加强大,更加完善,还需要很多工作要做。
本文仅仅在一个相对比较完善的嵌入式平台(树莓派)上完成了驱动的改写,从编译镜像文件与资源占用的角度对比了 ArceOS 操作系统与
一般嵌入式操作系统之间的性能差异。但还有大量的驱动对比还尚未涉及。
