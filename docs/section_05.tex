\section{结论}

本论文提供的成果作为 ArceOS 组件化操作系统驱动程序,在该平台建设中占有一定的地位。
本论文以几种常见嵌入式设备设备驱动为中心,首先构建了编译、调试 ArceOS 组件化操作系统的软件环境
,如交叉编译工具链的构建,Rust程序语言工具链的配置等前期准备。然后
重点分析了如 uboot, linux(rpi), circle 等操作系统在树莓派4B 上应用的以太网驱动,
了解了有关驱动程序的关键技术,如mdio,DMA等相关实现,
深入分析了网卡设备驱动程序的数据结构、驱动框架等,并基于 ArceOS 操作系统现有 Cvitek 驱动
,实现了与 ArceOS 操作系统嵌合的 BCM54213PE PHY 芯片驱动。还修改了现有串口设备驱动,在其之上添加了
对于 FPM383C 指纹识别模块通信协议的解析,在上位机中搭建了能自动上传指纹特征数据,下发给各指纹终端
和自动记录由下位机传输的打卡包的应用程序。

虽然本文的工作取得了初步的成功,完成以太网设备驱动的开发及相关软件编译,调试环境的构建,
但要使 ArceOS 组件化操作系统的功能更加强大,更加完善,还需要很多工作要做。
本文仅仅在一个相对比较完善的嵌入式平台(树莓派)上完成了驱动的改写,
虽然利用到了组件化操作系统所提供的条件编译带来的操作系统镜像大小优势,但是并没有
尝试利用这一优点\footnote{目前利用到这个优势的地方仅有通过串口传输二进制镜像文件,
ArceOS 编译出的镜像文件有且仅有不到200kb,相较于一般嵌入式操作系统大小小很多}实现什么特别的操作,
也没有体现其成本优势
如在对硬件要求更为严格的嵌入式开发板上进行测试等。只是通过我的实现,论证了确实 ArceOS 系统是可以
用于嵌入式开发的,在开发难度上相对 Exokernl 等较低,但驱动重写实际上工作量不低,可能对于现有设备
厂商而言,开发成本不太经济。

在实现驱动的过程中面临着比较多的问题,比如说由于 BCM54213PE 这款 PHY 芯片对应的底层信息尚未开源,无法通过查阅手册的方式
解决开发过程中遇到的比如说链路状态未正常建立的问题。同时,由于转写的时候不了解具体实现细节,大量的代码均依照树莓派 linux
内核通过类似于宏的 Rust 硬件抽象实现。在这种情况下,大量寄存器的操作是通过对绝对地址进行 volatile 读写方式实现的,
可能在其他开发板上进行复用的时候会面临一定程度上的困难。