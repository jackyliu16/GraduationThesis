\section{游戏重要系统设定}

\begin{figure}[ht]
    \centering
    \includegraphics[scale=0.5]{imgs/游戏逻辑.png}
    \caption{游戏逻辑}
\end{figure}


\subsection{总体概述}
玩家拥有一个城镇作为自己的发展基地。但初始城镇功能有限,玩家需要通过不断的前往森林地下城探索资源来提升特定建筑的等级来获取更有用的道具和装备,为自己之后的冒险提供便利。

\textbf{城镇光亮系统:}
城镇中心有座灯塔,灯塔需要燃料来维持自己的照明范围。同时玩家可以提升灯塔等级或相关科技使灯塔扩大照明范围,但同样的,玩家需要更多的资源来维持其更大的照明范围。照明范围影响着建筑系统与森林地下城的难度。当其达到一定程度时可以解锁新的建筑与科技,同时玩家可以选择更深处的森林地下城进行探索。但需要注意的是,当燃料不足以维持灯塔亮度时,照明范围会缩小。带来的结果是照明范围外的建筑会重新被黑暗吞噬,甚至可能降低已经提升的建筑等级。

\subsection{探索系统概述}

\subsubsection{探索光亮系统}
玩家初次进入地下城时仅仅有3根火把(数量及照明设备可以通过提升城镇规模来进行升级),每根火把可以燃烧120秒,也就是2分钟。光亮也会在每个照明设备的3/3,2/3,1/3,0/3时分别进行“亮——>较亮——>微亮——>黑暗”的模糊渐变。根据亮度的变化会出现不同强度的敌人对玩家进行攻击,不同强度的敌人也会掉落不同强度的装备与资源。

\subsubsection{关卡系统}
玩家在进入森林地下城时,游戏会随机生成“种子”,并根据种子生成地下城以达到Rogue Like游戏的目的。我们并未采用一次性生成整张地图的游戏方式,而是采用了更为大众化的碎片式地图探索方式进行游戏。每一层都有一个BOSS镇守,玩家需要打败BOSS才能进入下一层地图进行探索。

\begin{figure}[ht]
    \centering
    \includegraphics[scale=0.64]{imgs/《以撒的结合》BOSS战.jpg}
    \caption{《以撒的结合》中惊险刺激的BOSS战}
\end{figure}


同时关卡中会随机生成补给点与商店等,供玩家进行短暂的休整以降低游戏的难度,玩家可以在补给点或者商店重新点亮火把或是购买火把进行重整。每一层的初始点与BOSS点都设有“火焰传送门”,玩家可以通过传送门返回城镇来结束本次冒险。但要注意,通过初始点的传送门进行离开时会给予玩家包括资源折半,愧疚debuff等一定惩罚。


\subsubsection{随机事件}

在探索的过程中会出现随机事件,可能是更强大的敌人出现,也可能遇到其他冒险者的尸体,可能遇到重要NPC,甚至是进入了一个失落的遗迹等,为游戏添加更多的探索性。

\begin{figure}[ht]
    \centering
    \includegraphics[scale=0.59]{imgs/突发事件.jpg}
    \caption{游戏中的突发事件}
\end{figure}