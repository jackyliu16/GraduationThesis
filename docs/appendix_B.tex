% \nonumsection{附录 - 测试记录}

% \begin{table}[ht]
%   \centering
%   \caption{帧大小与传输速率测试表}
%   \pgfplotstabletypeset[
%     col sep=comma,
%     columns={EthPackSize, AvgBandwidth},
%     header=true,
%     every head row/.style={before row=\toprule, after row=\midrule},
%     every last row/.style={after row=\bottomrule},
%     % 设置读取行数
%     read from file row filter/.code={
%         \pgfplotstablerow<12 % 只显示前12行
%     }
%   ]{./imgs/测试记录.csv}
% \end{table}

% \begin{table}[ht]
%   \centering
%   \caption{传输延时与传输速率测试表}
%   \csvreader[
%     longtable={llll},        % 表格的列格式
%     table head=\toprule 传输延时 & 传输速率(128) & 传输速率(376) & 传输速率(512)\\\midrule,   % 顶线和中间线
%     late after line=\\,   % 每行结束
%     table foot=\bottomrule % 底线
%   ]{./imgs/测试记录.csv}{}{\csvcoliii & \csvcoliv & \csvcolv & \csvcolvi }
% \end{table}

\nonumsection{附录 - 测试记录}

\begin{table}[ht]
  \centering
  \caption{Linux 驱动传输速率测试}
  \label{tests::Linux驱动传输速率测试}
  \begin{adjustbox}{width=1\textwidth}
    \csvautobooktabular{./imgs/测试记录-测试脚本Linux测试.csv}
  \end{adjustbox}
\end{table}

\begin{table}[ht]
  \centering
  \label{tests::ArceOS驱动传输速率测试}
  \caption{ArceOS 传输速率测试}
  \begin{adjustbox}{width=1\textwidth}
    \csvautobooktabular{./imgs/测试记录-测试脚本ArceOS测试.csv}
  \end{adjustbox}
\end{table}

% \begin{table}[ht]
%   \centering
%   \caption{帧大小与传输速率测试记录表}
%   \csvautobooktabular{./imgs/测试记录-传输率与以太网帧.csv}
% \end{table}

% \begin{table}[ht]
%   \centering
%   \caption{帧大小与传输速率测试表}
%   \begin{tabular}{llll}
%     \toprule
%     Delay & Bandwidth \& 128 Size & Bandwidth \& 376 Size & Bandwidth \& 512 Size \\
%     \midrule
%     % 读取 CSV 文件
%     \csvreader[
%       late after line=\\, % 每行结束
%       head to column names
%     ]{./imgs/测试记录-传输率与传输延迟.csv}{}{\csvcoli & \csvcolii & \csvcoliii & \csvcoliv}
%     \bottomrule
%   \end{tabular}
% \end{table}