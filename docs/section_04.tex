\section{设计结果论述与不足思考}

    指纹考勤系统以树莓派4B(BCM2711)为核心,包括以下组件:
    指纹传感器HLK-FPM383C、
    基于树莓派板载BCM54213PE PHY芯片的通信模块,
    以及由上位机提供的数据存储,日志记录的部分。

    全文主要完成了以下六个方面的工作:

    \begin{itemize}
        \item 完成了指纹考勤系统整体设计,对于一个最简易的单片机指纹考勤系统进行了需求分析。
        \item 完成了指纹考勤系统硬件电路设计。主要包括主控模块与信号回返模块,串口输入模块之间的硬件电路设计。
        \item 在组件化操作系统 ArceOS 上仿照树莓派官方 linux 内核中驱动相关实现,实现并整合了一个能运行的以太网驱动,
            该驱动实现了以太网物理层与数据链路层之间的通信,
            嵌入式裸机应用程序可以深度集成到ArceOS中,调用由 ArceOS 实现的 Tcp/Udp 以太网协议栈以在应用层面调用 UdpSocket 所提供的服务编译成一个完整的单一用途裸机应用程序。
            通过这个以太网驱动,可以实现下位机与上位机之间的以太网帧通信\ref{fig::树莓派通信正常}。
        \item 将上述以太网驱动与 ArceOS 中提供上层网络包装的 axnet module\footnote{基于smlotcp的实现} 进行结合,
            实现了基于 UdpSocket 的发包功能。
        \item 在组件化操作系统 ArceOS 中,基于自建 UART 串口驱动与FPM383模组通信协议\cite{noauthor_fpm383c_nodate},
            实现了 ArceOS 的指纹模块,能基于串口信号传输,实现简单的如LED亮灯,要求FPM383C模块进行指纹匹配,
            对于收到的指纹匹配结果进行分析等功能。
        \item 根据上述指纹考勤系统设计需求,实现了一个基于 ArceOS 改写标准库的裸机应用程序。
    \end{itemize}

    在实现的过程中面临着比较多的问题,比如说由于 BCM54213PE 这款 PHY 芯片对应的底层信息尚未开源,无法通过查阅手册的方式
    解决开发过程中遇到的比如说链路状态未正常建立的问题。同时,由于转写的时候不了解具体实现细节,大量的代码均依照树莓派 linux
    内核通过类似于宏的 Rust 硬件抽象实现。在这种情况下,大量寄存器的操作是通过对绝对地址进行 volatile 读写方式实现的,
    可能在其他开发板上进行复用的时候会面临一定程度上的困难。

    实现的不足之处:

    \begin{itemize}
        \item 我在具体实现以太网卡驱动的时候,部分操作还没有调通,从 BCM54213PE PHY 芯片所返回的状态检查一直
        显示当前链路状态未建立,查阅了全部可能有关的状态寄存器,仍然无法明确具体这种情况是由什么原因导致的。
        幸运的是,在深入研究时我发现了一个名为 rpi4-rust-workspace 的代码仓库 \cite{rpi4-rust-workspace},
        在这个最后更新于20年的仓库中提供了一个相较于我实现更为完善的以太网卡驱动,只不过由于仓库维护者太久没有进行维护,
        在仓库中所引用的很多 rust 语言的方法没有办法在 ArceOS 所支持的 rust 工具链中正常运行。
        同时,仓库中也存在着很多由于维护者直接将 rpi3 版本中寄存器复制到 rpi4 版本的过程中没有详细检查寄存器等调用是否一致所导致的问题,比如说在
        UART 所牵涉到的 GPIO 部分\footnote{根据FPM383C模块说明,必须要在Rx设定上拉电阻},维护者直接将在
        rpi3(bcm2835) 使用的 GPIO Pull-up/down Register 也就是 GPPUD 寄存器等规约复制到了 rpi4(bcm2711)
        中。\cite{raspberry-pi-bcm2835}但是,树莓派 4 中采用了另外的一组寄存器来控制上拉下拉电阻 \cite{raspberry-pi-bcm2711}
        GPIO\_PUP\_PDN\_CNTRL\_REG0,二者类型不同,起始地址也不同。
        亦或是由于作者实现串口驱动的目的在于通过串口反馈简单字符串,并没有考虑到在大信息通量下的数据获取问题,因此没有启用 FIFO 等。
        \item 由于时间因素的影响,并没有如同前面设计的那样,实现了多种不同的上位机,下位机通讯机制,只完成了基于 BCM54213PE 板载
        PHY 芯片的网络通信机制实现。
    \end{itemize}