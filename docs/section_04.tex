\section{设计结果论述与不足思考}

    指纹考勤系统以树莓派4B(BCM2711)为核心,由指纹传感器HLK-FPM383C,基于树莓派板载 BCM54213PE PHY 芯片的通信模块,
    上位机提供的数据存储模块等组成。全文主要完成了以下四个方面的工作:

    \begin{itemize}
        \item 完成了指纹考勤系统整体设计,对于一个最简易的单片机指纹考勤系统进行了需求分析。
        \item 完成了指纹考勤系统硬件电路设计。主要包括主控模块与信号回返模块,串口输入模块之间的硬件电路设计。
        \item 在组件化操作系统 ArceOS 上基于模仿树莓派官方 linux 内核中驱动相关实现,实现并整合了一个能运行的以太网驱动,
            该驱动实现了以太网物理层与数据链路层之间的通信,允许嵌入式裸机应用程序能够通过与 ArceOS 深度链接,编译成为一个
            完整的,单一用途的,裸机应用程序。
        \item 在组件化操作系统 ArceOS 中,基于自建 UART 串口传输驱动与FPM383模组通信协议\cite{noauthor_fpm383c_nodate},
            实现了 ArceOS 的指纹驱动。
        \item 根据上述指纹考勤系统设计需求,实现了一个基于 ArceOS 改写标准库的裸机应用程序。
    \end{itemize}

    在实现的过程中面临着比较多的问题,比如说由于 BCM54213PE 这款 PHY 芯片对应的底层信息尚未开源,无法通过查阅手册的方式
    解决开发过程中遇到的比如说链路状态未正常建立的问题。同时,由于转写的时候不了解具体实现细节,大量的代码均依照树莓派 linux
    内核通过类似于宏的 Rust 硬件抽象实现。在这种情况下,大量寄存器的操作是通过对绝对地址进行 volatile 读写方式实现的,
    可能在其他开发板上进行复用的时候会面临一定程度上的困难。

