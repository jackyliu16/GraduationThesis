% 完整编译: xelatex -> bibtex -> xelatex -> xelatex
\documentclass[UTF8,AutoFakeBold=1,AutoFakeSlant,zihao=-4]{cucugthesis}

\usepackage{enumerate}


% 在这里填写你的论文题目
\newcommand{\thesisTitle}{如何使用{\LaTeX}生成本科毕业论文}  % 论文的中文名称
\newcommand{\thesisTitleEN}{Generation of an Undergraduate Thesis Using {\LaTeX}}  % 论文的英文名称

% 在这里填写你的相关信息
\newcommand{\yourDept}{计算机与网络空间安全学院}
\newcommand{\yourMajor}{计算机科学与技术}
\newcommand{\yourName}{刘逸珑}
\newcommand{\yourNameEN}{Y. Zhang}
\newcommand{\yourClass}{2017级计算机科学与技术}
\newcommand{\yourMentor}{周老师}
\newcommand{\studentID}{201720111222333}

\begin{document}
% 封面(自动生成)
\coverpage

\begin{abstract}
    % TODO check
    \fontsize{15}{20}\selectfont
    随着信息化时代的逐渐发展,各种芯片企业由于芯片制程的缘故,在开发体积更小,更为集中,的芯片上面临了越来越多的问题。
    本文尝试采用一种实验性的操作系统 arceos,利用其类同于 libos
    的优秀组合能力,间接利用rust语言零抽象成本的效果,将现有的,由设备厂商提供的原始设备驱动,转化成为 arceos 上能使用的驱动,并尝试将其与一般 RTlinux, RTOS 如 RTThread 等操作系统剪枝之后进行对比,尝试讨论采用这种新型的操作系统重写现有实现的优劣。
    \keywords{unikernel, arceos, 组件化操作系统,网卡驱动}     % 中文关键词
\end{abstract}

% 英文摘要
% 英文摘要
% \begin{abstractEN}
% With the coming of the era named ``Third Generation'',
% the mobile communication technologies are changing
% its direction from providing simple services of voice
% and low speed data transmission to providing combination
% services of voice and multi-media data transmission with
% variable band-width. Antennas, as the most important part
% in the radio link of mobile communication, become an
% important research topic, and attract more and more attentions.
% \keywordsEN{DSP, Spectral Analysis, FFT, MATLAB}       % 英文关键词
% \end{abstractEN}

% 目录(自动生成)
\contentpage

% \section{绪论}
这个部分直接从开题报告上面抄下来的,还没改

\subsection{项目背景及意义}
TODO 重新引用,检查内容是否合适

今年来,由于一系列计算机网络安全事件的出现,操作系统领域的安全性逐渐被国家重视。
无论是2017年知名闭源操作系统windows中爆出的 永恒之蓝 病毒,还是近年来伊朗核试验基地遭遇“震网”病毒袭击时间,亦或是最近这段时间 linux 操作系统所面临的开源软件 xz 源供应链投毒事件,无不告诉我们操作系统领域安全性的重要性。
作为国民生产生活所不可或缺的一部分,一旦我们没有办法能保障操作系统的安全性,个人计算机尚且可能会被窃取隐私数据,大型的科研计算机亦或是国防科工设备一旦由于操作系统上的后门而遭到入侵,必然会导致巨大的损失。
因此,操作系统作为现代化行业的重要生产工具,其安全性,稳定性和可靠性必然需要得到保障。

而随着 rust 这种新型的,通过强制类型类型声明,所有权检查机制,生命周期检查器等一系列方式保障安全性的编程语言的出台,逐渐有使用此类编程语言进行操作系统设计实现的实验。如 (LANKES 等, 2019) 就在其论文中剃刀了
使用 rust 高级语言代替在 unikernel 中进行开发常用的 c 语言的设想。甚至在后来(LANKES 等, 2020) 尝试将其最初的 HermitCore 重新通过 Rust 语言进行了相关的实现 (Rusty Hermit),结果证明 Rust 语言实现版本与 C 语言实现版本之间
不存在性能差异。同时,由于 Rust 独特的标记不安全代码的设定,在操作系统中有且仅有 3.27% 的代码的安全性需要特殊注意。基于类似于 (BOOS 等,2020) 提出的 Theseus 操作系统,与 (KUENZER 等, 2021) 等提出的 Unikraft 微库操作系统实现理念的理解
,贾越凯博士实现了一种将组件化与库操作系统结合在一起的,基于 Rust 安全性保障的新型操作系统 ArceOS。

这种操作系统类似于较早年间外核(ExoKernel)的设计类似,由于其主要设计在嵌入式平台等专用平台上进行使用,而非基于某种通用的硬件设备或者 VMM 虚拟机的尝试,在实现层面上面临了许多硬件层面的适配问题。

本次毕业设计主要是想要通过在这种不同于传统操作系统架构,如宏内核或者微内核的类 LibOS,Unikernel 架构下操作系统中实现一些简单嵌入式设备常见驱动的过程,为今后的学习打下坚实的基础,也同时为我国现代化操作系统的发展实现添加我的一份力量。

% \subsection{国内外研究现状}

\subsubsection{Unikernel 国内外研究现状分析}

出于成本考量进一步加剧。在功能复杂的操作系统,比如说 Linux 操作系统上运行轻量级服务会由于通用操作系统所进行的必要抽象
支付大量的开销(ANDERSON 等)(MASSALIN 等, 1989)。因此,找到某种方法以细粒度的定制操作系统是存在必要的。也因此 (ENGLER 等) 通过
一种名为 Secure Bingdings 的机制,将 LibOS 与其他用户库绑定在一起,允许应用程序以更加具体到硬件的方式对于底层硬件进行调用,
释放其性能而不受操作系统层面抽象的局限。但是这种实现方法由于机制内部会导致额外的上下文切换,同样会带来不必要的成本。
[ PESSÉ S, XIA Y, QIU L, 2015. 计算机系统原理讲义[EB/OL]//计算机系统原理(课程讲义). (2015-07-21)[2023-12-06]. https://unitial.gitbooks.io/csp/content/index.html.]。
同时,从底层硬件设备驱动的角度上来讲,一般这些设备只会针对于部分通用操作系统进行适配,由操作系统层面完成这类支持是没有效率的。

不过,由于云服务商对于不同环境中一致性和可移植性的需求,类似于 docker 这种容器化技术逐渐出现,(MADHAVAPEDDY 等) 基于容器技术或者
虚拟机管理程序提供的底层一致性平台上,提出了 Unikernel 这种专用的,单地址空间的使用库操作系统理念构造的组件化操作系统
,这种构建方式解决了如 ExoKernel 等操作系统所面临的,在底层硬件上的兼容性问题 ,提供了一个小型的,安全的,快速的工作负载。
同时,由于该操作系统在实现类似于一般操作系统的进程管理等功能外不提供额外如 Shell 等功能,
降低了由于外界侵入所产生的危害与攻击面,在实现了更大程度上的安全保障的同时,也使系统能够更加紧凑
\footnote{相较于一般的BIND DNS的Linux VM 镜像达到462MB,
根据他们方法生成的Mirage设备生成的镜像大小只要183.5kB
(没有实现所有功能集,但是包含了queryperf测试套件的必要功能,且可用于自托管项目)]的高度专业化单一用途设备虚拟机。 }
,根据用户需求进行客制化。

(KUENZER 等, 2021) 在他们的研究中提出了一种名为 Unikraft 的新型微库操作系统,这种操作系统通过其特殊的编译,链接系统
提供了完全模块化的操作系统基元和一系列高性能 API 以方便用户对于操作系统进行定制。不同于前者基于高度定制 c 语言程序链接实现的思路,
(BOOS 等,2020) 的开发实践中通过其对 Rust 高等程序语言的编译编译工具链的利用实现了类似的效果,他们提出了一种名为 Theseus 的操作系统
这种操作系统通过减少一个组件为另一个组件保留的状态来重新设计和改进操作系统模块化,在尽可能的将保障内核安全的工作转嫁给 Rust 编译器
进行实现的基础上实现了对于核心操作系统组件的实时演化与故障恢复。不过,由于其代码特性,该操作系统实现的颗粒度相对更大,
不利于一般开发者参与进行开发。
贾越凯博士提出 ArceOS 的在其基础上粗放了对应核心组件的颗粒度,使得大部分的操作系统组件都可以以更加合乎一般操作系统开发者的开发
方式进行实现,在降低开发难度的同时,也实现了安全保障。


% (4)(PORTER 等) 在他们的研究中通过将广泛使用的单片操作系统(Windows 7)重构为一个功能丰富的库操作系统(Library OS),
% 由一个小型抽象集(线程,虚存,I/O流)连接 Library OS 和 Host OS实现了独立于底层内核组件。
% 在早期的 Library OS 支持者中认为,通过个性化定制每一个应用程序,可以实现OS的较好性能,
% 但是现在Library OS已经成为了现代虚拟机监视器的牺牲品。 作者在本文中通过将每个Library OS的特性(应用程序依赖的) 共享底层的 
% Host OS 资源,这样在操作中,只需要根据应用程序的API进行提取,这样相较于虚拟化整个OS会大大的降低开销。
% % \footnote{\href{source}{https://blog.csdn.net/qq_40119224/article/details/118754160}}

\subsection{研究目标}

1、研究的主要内容
本次研究主要分为以下几个目标,搭建嵌入式环境,编写指纹识别驱动以及网卡驱动实现。

2、研究的预期目标
能够在同一网段下实现或者更小的嵌入式开发板的相互通信,并且能够在Client端获取输入的指纹信息,将其加密后传输到Servers段进行解密,在解密之后Servers段传送解密成功信息给Client端,Client端通过显示屏显示对应信息或者通过指示灯表示打卡成功的信息。

3、研究的创新点
    尝试通过一种新颖的组件化操作系统对于一个简单应用场景进行重构。

\subsection{研究方法}

首先先简单完成电脑中有关于Rust等内容的开发环境配置,同时根据Nix生成一套可重构的开发环境框架,确保在后面的开发过程中整体开发过程是可复现的。
其次,深入了解现有的网卡驱动开发经验,并且了解其中所遇到的问题以及可能性的解决方案。在此基础上对于As608指纹识别模块驱动的现有实现进行了解,了解其中主要使用的数据结构等内容,以方便后面通过rust语言对于现有驱动进行重构。
然后,通过对于现有的指纹识别系统的分析,了解到目前常用的指纹识别框架。尝试了解no\_std情况下以及std情况下开发的差异。同时分析可能需要用到的库大概有多少是需要重新实现或者替换库的。
通过ArceOS现有的工作完成网络联网测试,实现几个简单的用户程序,首先确保能在多台华山派CV1811H甚至支持更少功能的嵌入式设备上运行简单发包程序并且联网成功。根据已有驱动尝试在ArceOS上进行测试等开发工作,最终将指纹识别驱动以及可能会用到的显示屏驱动,装载上进行整机测试。

引用图\ref{fig},表\ref{tab},公式\eqref{eq:0}
  % 第一章,在保存.tex文件时一定使用UTF-8编码。
% \section{系统设计}

\subsection{需求分析与模块选型}

根据一般企事业单位对于考勤事务的管理规范,本指纹考勤系统需要实现以下几种功能,通过上位机对于下位机中指纹识别模块保存的指纹信息进行注册与删除,下位机基于前者提供的指纹数据实现基于光学识别的指纹打开签到功能。

根据上述系统功能需求,本设计以树莓派 4B 嵌入式开发板所提供的 bcm2711 作为中央处理芯片
,指纹考勤系统主要由电源供电模块,声音反馈模块,USB转TTL串口通信模块,网卡模块。
系统总体设计方案如图\ref{总体设计图}所示。

% https://www.processon.com/v/65ec0156778cc21034664557
\begin{figure}[ht]
    \centering
    \includegraphics[width=\textwidth]{imgs/总体设计图.png}
    \caption{总体设计图}    \label{总体设计图}
\end{figure}

\subsubsection{嵌入式开发板选型}

树莓派 4B (如图\ref{树莓派硬件配置说明图}所示)使用的 bcm2711 是一颗四核心 64 位 ARM Cortex-A72 架构 CPU ,主频高,能满足多种复杂计算需求以及满足大型程序运行需求。
树莓派 4B 还存在丰富而完善的接口,两个 USB3.0 接口、两个 USB2.0 接口、一个千兆网卡接口、一个 HDMI 接口、一个 CSI 接口和一个 DSI 接口,能够满足对于各种外设的连接需求。
树莓派 4B 还是首款支持不通过 USB 接口而直接访问网卡芯片以实现网络连接的树莓派开发板,
这无形之中对于实现板载网卡驱动提供了很多帮助。

\begin{figure}[ht]
    \centering
    \includegraphics[width=\textwidth]{imgs/树莓派硬件配置说明图.png}
    \caption{树莓派硬件配置说明图}    \label{树莓派硬件配置说明图}
\end{figure}

\subsubsection{指纹识别模块选型}

FPM383F 是一款功耗低的光学指纹识别模块,采用半导体面阵传感器。
它可以存储60组光学指纹,通过串口与中央处理器进行通信。在串口驱动方面,模块可以通过树莓派的底层寄存器 UART 调用,以较容易地完成信息发送。

根据模块的规格书,指纹模块需要在上电后至少等待180毫秒才能正常通信。
在下电前,要将 MCU 的串口设置为输入高阻态\footnote{树莓派不支持将 GPIO 设置为高阻态},并且需要在Rx上添加上拉电阻,这些要求相对容易满足。

\begin{figure}[ht]
    \centering
    \includegraphics[scale=0.6]{imgs/FPM383C串口设备图.png}
    \caption{FPM383C串口设备图}    \label{FPM383C串口设备图}
\end{figure}


\subsubsection{通信模块选型}

    基于一般企事业单位对于考勤签到需求的需求,我计划提供多种不同的通信模块实现方便选用单位进行选择,
    其中传统基于 CH340 串口转 TTL 通信模块实现的简单串口通信主要适用于仅对于一两台设备进行支持的情况,
    而基于网卡模块间接通过网络方式拓展下位机数量的方式是主要计划实现的支持。
    针对于不同的预算管理需求,计划采用两种不同的方式实现网卡驱动,
    一种是基于 raspberry4B 板载 bcm54213PE 网卡芯片的驱动,
    另一种是基于 ENC28J60,一种基于 SPI 连接的外置 10BASE-T 以太网连接模块实现的。
    但是目前只实现了基于 BCM54213PE 网卡芯片驱动的支持。

    考虑到企事业单位对考勤签到的通用需求,我计划提供多种通信模块以便选用单位进行选择。
    一方面提供传统的 CH340 串口转 TTL 通信模块以适用于支持一到两台设备的简单串口通信。
    另一方面,我主要计划通过网络方式发送数据包以扩展下位机的数量。

    根据不同的预算管理需求,我计划采用两种方式来实现裸机网络通信支持。
    一种方式是实现 Raspberry Pi 4B 的板载 BCM54213PE PHY 芯片驱动;
    另一种方式则是采用 ENC28J60 ,这一种基于 SPI 连接的外置 10BASE-T 以太网模块。

\subsection{硬件设计}

    由于本实现相对来说比较轻量化,硬件上不太需要外部模块的支持,只需要通过杜邦线连接各模块即可(如图\ref{外部模块接线图}所示)。

    \begin{figure}[ht]
    \centering
    \includegraphics[scale=0.6]{imgs/接线图.png}
    \caption{外部模块接线说明图}    \label{外部模块接线图}
    \end{figure}

    \begin{itemize}
        \item 信号回返模块:
            按照最初的设计是采用额外的信号提示模块,如基于树莓派 GPIO 电信号控制实现的蜂鸣声或者 LED 闪烁提示当前已经完成
            打卡,但是这部分最后通过 FPM383C 模块内部的 LED 控制信号完成了提示操作。
        \item 指纹采集模块:
            本模块直接采用了现有的 FPM383C 指纹信号采集模块,直接通过调用树莓派串口服务,在原先串口通信模块的基础上实现
            对于指纹采集模块的通信。
            在硬件层面上,直接采用端子线将树莓派与 FPM383C 模块连接,连接示意图如\ref{外部模块接线图}右侧所示\cite{fpm383c-module-specification}。
        \item 网络通信模块:
            本模块主要使用了现有的树莓派板载 PHY 芯片 BCM54213PE 实现了对应 PHY 层信号解析,底层通过 RJ45 水晶头以
            双绞线直接连接笔记本网口。计划中额外支持 ENC28J60 模块还需要额外通过 GPIO 7,8,9,10,11 Pins 实现 SPI 端口连接
            来实现类似于树莓派板载 PHY 芯片的效果。
        \item 串口通信模块:
            在本次嵌入式实现中直接采用现成的 TTL 转 USB 模块 CH340 实现对于树莓派串口通信读取,具体接线方式如 \ref{外部模块接线图} 左侧所示,
            具体实现上通过被装载在树莓派上的系统中所默认打开的 UART0 串口(Pins 14,15)实现基本的信息传输
            \footnote{在后文中有指出,目前出于调试方便的目的,
            通过 rust 编译的裸机二进制文件会在树莓派上电发送信号(由SD卡上装载的默认内核实现)与前台 minipush 同步之后,
            经由串口被装载到设备内存中,最后默认内核会转交控制权给 ArceOS 的起始地址}
    \end{itemize}

\subsection{软件设计}

    由于本研究所采用的嵌入式应用场景相对单一,且仅涉及单应用程序单地址空间,
    % 由于本研究所采用的基础嵌入式应用程序所面对的的嵌入式应用场景相对较为单一,同时也是单应用程序单地址空间的。
    \footnote{虽然底层操作系统支持使用页表进行隔离,但是在应用层面上并没有使用到虚拟页表,只是在boot的时候使用了内核页表}
    因此整体软件设计相对较为简单,主要实现难度在驱动设计层面体现,具体内容与开发过程在第三章中进行呈现。
    在嵌入式应用中,系统通过循环读取UART串口设备的数据反馈,并通过ArceOS封装的底层以太网驱动,
    将分析后的结果根据需求封装成 UDP 包\footnote{保存打卡信息等},通过 RJ-45 接口发送至上位机的管理应用程序

    在上位机中通过简易的 python 客户端程序,对于嵌入式设备中传输的 UDP 包进行分析,实现基于 SQLite 数据库的简单指纹信息 CRUD\footnote{增删改查},打卡数据 CRUD,
    与一个基于命令行实现的打卡记录查询与导出应用程序。

    \subsubsection{嵌入式软件设计}

    参照一般嵌入式考勤打卡设备的标准结构,我设计的嵌入式应用在完成对应变量初始化之后维持在一个主循环(如附录代码段\ref{algorithm::fingerprint_network_comm}所示)中运行。
    一次循环包括收网络包\footnote{匹配帧头后向指纹模块下发对应UART包实现下载,删除指纹特征的效果}
    ,执行指纹匹配命令,对于指纹匹配结果进行查询,并在完成匹配时通过网络向上位机以特定的帧形式发送网络包告知其
    已经收到用户的打卡信息。

    \begin{algorithm}[htb]
        \caption{嵌入式设备主循环}
        \label{algorithm::fingerprint_network_comm}
        \begin{algorithmic}[1]
        \While{true} \Comment{嵌入式设备主循环事件}
            % \State \textbf{try} \Call{ReceiveData}{$socket, \& buf$}
            \If{\Call{Receive}{$socket, \& buf$} 正常返回 $(size, addr)$} \Comment{收到了网络包}
                \If{收包结果与特定帧头匹配}
                    \State \Call{Write}{serial, \text{\&buf[..size]}} \Comment{将帧转发给从属指纹模块}
                \EndIf
                \State $buf \gets [0; 1024]$ \Comment{重置缓冲区}
            \Else{打印错误信息}
            \EndIf
        
            \State \Call{Write}{serial, \text{search\_fingerprint\_match\_pattern}} \Comment{要求设备自动匹配指纹}

            \State \Call{Delay}{10} \Comment{等待模块进行指纹匹配}
            \State \Call{GetFrame}{$serial$} \Comment{处理返回帧}
            
            \State \Call{Write}{serial, \text{check\_match\_\_result\_pattern}} \Comment{查询匹配结果}
        
            \If{$frame \gets$ \Call{GetFrame}{$serial$}}  \Comment{对于返回包进行分析}
                \State \Call{Assert}{$frame$, CmdType::CheckMatchFingerprint} \Comment{确保应答包一致}
                
                \If{\Call{Any}{$data, \neq 0$}}
                    \State \textbf{initialize} $sign\_in\_buf \gets [0x46, 0x69, \ldots, 0x20]$ \Comment{网络包}
                    \State \Call{SetSignInData}{$frame, sign\_in\_buf$} \Comment{根据应答包设置 buf 中与ID部分}
                    \State \Call{SendTo}{$local\_socket, sign\_in\_buf, target\_socket$}
                    \State \Call{Write}{serial, \text{green\_flashing\_pattern}} \Comment{向设备发送绿灯闪烁}
                \EndIf
                \State \Call{GetFrame}{$serial$}
            \EndIf
        \EndWhile
        \end{algorithmic}
        \end{algorithm}
        

    \subsubsection{数据库设计}

    由于对应本次嵌入式设计的上位机中的操作需求相对较为简单直白。同时,不同于课堂考勤系统等需要额外的提供很多
    诸如教学班,教室,教师等信息的需求,需要维持一系列诸如教学班表,教师表,学生表等\cite{基于WiFi探针的智能考勤系统设计}。
    本次嵌入式软件在功能层面仅需要完成简单的考勤打卡记录,指纹特征模块的存储
    的功能,因此在表实现上仅需要维持三个表的存在,分别是指纹特征表,指纹信息表,员工信息表,考勤打卡表
    。

    因此,在这样的一个简单的的嵌入式项目中使用诸如 Oracle, MySQL, PostgreSQL 等大型企业级数据库是没有效率
    的行为。在设计中综合考量了一系列如 SQLite, LMDB 等嵌入式常用的数据库,考虑到常用性,库支持程度等因素之后
    选用了 SQLite 作为嵌入式信息存储的数据库。

    \begin{figure}[ht]
    \centering
    \includegraphics[scale=0.6]{imgs/数据库关系图.jpg}
    \caption{数据库关系图}    \label{fig:db-total}
    \end{figure}

    数据库整体设计如图\ref{fig:db-total}所示,主要由四类表构成,从左上到右下分别为
    打卡信息表,指纹信息表,用户信息表,指纹特征表。

    \begin{description}
        \item[用户信息表] 存储常见的用户信息数据

            在用户信息表\ref{tab:userInfo}中完成对于用户ID,
            指纹特征信息的存储属于一种最基本的考勤系统嵌入式设计需求。
            在本表中我将 user\_id 作为主键,唯一性的声明用户的身份,
            同时,我并没有将指纹 ID 作为外键保存在此表中方便进行级联删除,
            主要是考虑到用户与指纹 ID 之间实际上不是绝对的一一对应关系。

            \begin{lstlisting}[language=SQL
                , caption={用户信息表}
                , label = {tab:userInfo}
                , breaklines=true
                , breakatwhitespace=true]
CREATE TABLE IF NOT EXISTS user (
    user_id INTEGER PRIMARY KEY AUTOINCREMENT,
    user_name TEXT NOT NULL,
);
            \end{lstlisting}
        
        \item[指纹信息表] 存储指纹 ID 与 用户 ID 之间关系,还有对应指纹特征信息长度的关联表。
        
            此处将指纹 ID 与用户 ID 之间的联系单独抽象出来设计表主要是考虑到指纹与用户之间并不是强制的一一对应关系,
            因此不能直接将二者合并到同一张表中。
            同时由于在实际发送的时候指纹模块通信协议要求先发送指纹特征长度,然后再分帧发送数据,
            直接将这个指纹特征长度信息存储在指纹特征表中无疑是带来了很多浪费,因此引入了 data\_len 作为参数。

            \begin{lstlisting}[language=SQL
                , caption={指纹信息表}
                , breaklines=true
                , breakatwhitespace=true]
CREATE TABLE IF NOT EXISTS FingerInfo (
    finger_id INTEGER PRIMARY KEY,
    user_id INTEGER,
    data_len INTEGER,
    FOREIGN KEY (user_id) REFERENCE user(user_id) ON DELETE CASCADE
);
            \end{lstlisting}          

        \item[指纹特征表] 存储帧号,二进制指纹特征信息等内容。
        
            根据 FPM383C 用户通信协议,在指纹特征信息上传过程中,每一帧传输除最后一帧外,均只发送128位的指纹特征数据
            \cite{fpm383c-modular-communication-protocol}。
            同时,在将指纹特征信息下载的过程中也存在类似的要求,因此在实现考量上,先在上位机中接收数据包
            合并成为一个单独的二进制数据包存储到数据库中,同步下载的时候将整个数据包切分成为不同块再发送不仅
            会导致对上位机中更高的内存空间占用\footnote{SQLite以页的形式存储数据},还会增加很多无意义的工作量。
            因此,我们选择不使用整体指纹特征存储的方法,而是直接在上位机接收到的数据中,
            根据每个数据包自带的 frame 号对数据进行分割存储,以简化处理过程。
            这样在上传和下载一个数据包的时候,就只需要对于数据库中的一行进行处理,降低了很多工作量,同时也简化了数据包丢失的处理步骤。

            同时,由于在实现过程中方便调试管理和降低内存空间无意义占用的需求,指纹特征表并没有使用直接在一个表中
            类似于 (finger\_id, frame\_num, data\_len) 的结构来存储指纹特征信息。
            而是根据指纹 ID 关联到不同的数据表中,在数据表中以 frame\_num 作为主键,存储二进制特征数据。
        
            \begin{lstlisting}[language=SQL
                , caption={指纹特征表}
                , label = {tab:fingerprintInfo}
                , breaklines=true
                , breakatwhitespace=true]
CREATE TABLE IF NOT EXISTS Finger{id}Data (
    frame_num INTEGER,
    data BLOB NOT NULL,
);
            \end{lstlisting}          


        \item[考勤打卡表] 记录考勤打卡的时间信息
        
            在考勤打卡表\ref{tab:signIn}中主要保存对应于 user\_id 的打卡数据,其中通过默认CURRENT
            \_TIMESTAMP的方式自动插入了对应的时间戳数据,方便用户进行插入操作。

            \begin{lstlisting}[language=SQL
                , caption={考勤打卡表}
                , label = {tab:signIn}
                , breaklines=true
                , breakatwhitespace=true]
CREATE TABLE IF NOT EXISTS signIn (
    user_id INTEGER,
    time TEXT DEFAULT CURRENT_TIMESTAMP
)
            \end{lstlisting}    

    \end{description}

    \subsubsection{功能说明}

    \begin{description}
        \item[指纹注册] 由上位机中自动完成指纹注册,通过上传命令,获取对应ID的指纹特征信息并以UDP包的形式下发到下位机中的指纹模块。
        
        指纹注册功能主要在上位机中完成,这主要是考量到在 HR 处实现人事登记等操作
        更加合乎一般企业的考勤流程。
        
        通过 0x01 0x18 命令(见表\ref{uart::auto-register}),实现自动注册功能,该命令会自动完成采图、提取、拼接、保存等操作,
        该部分通过分析返回包中的ID和注册进度进行操作,在注册进度达到 0x64 时终止注册流程。

        \begin{table}[htbp]
            \caption{自动注册命令用户层帧} \label{uart::auto-register}
            \resizebox{\textwidth}{!}{%
                \begin{tabular}{llllllll}
                \hline
                \multicolumn{1}{c}{校验密码} & CMD类型 & CMD号 & 等待手指 & 按压次数 & ID\_H & ID\_L & 校验和  \\ \hline
                0x00 0x00 0x00 0x00        & 0x01  & 0x18 & 0x01 & 0x06 & 0xFF  & 0xFF  & 0xE2 \\ \hline
                \end{tabular}
            }
        \end{table}

        在注册完成之后,通过上载命令(见表\ref{uart::upload-info}),向指纹模块获取特定 ID 号的指纹特征信息长度,
        然后再通过 (见表\ref{uart::upload-data}) 命令,从指纹模块获取对应分片的指纹特征信息。
        在将信息存储到数据库的同时,还通过 UDP 包的形式,将对应的数据发送到树莓派,树莓派再将对应指纹特征信息
        下载到树莓派对应的指纹模块上,由此完成了一次标准的指纹注册功能。

        \begin{table}[htbp]
            \centering
            \caption{获取上传信息命令用户层帧} \label{uart::upload-info}
            \begin{tabular}{llllll}
            \hline
            \multicolumn{1}{c}{校验密码} & CMD类型 & CMD号 & ID\_H & ID\_L & 校验和  \\ \hline
            0x00 0x00 0x00 0x00        & 0x01  & 0x53 & 0x00  & 0x01  & 0xAB \\ \hline
            \end{tabular}
        \end{table}

        \begin{table}[htbp]
            \caption{获取指纹特征命令用户层帧} \label{uart::upload-data}
            \resizebox{\textwidth}{!}{%
            \begin{tabular}{llllllll}
                \hline
                \multicolumn{1}{c}{校验密码} & CMD类型 & CMD号 & ID\_H & ID\_L & NUM\_H & NUM\_L & 校验和  \\ \hline
                0x00 0x00 0x00 0x00        & 0x01  & 0x51 & 0xFF  & 0xFF  & 0x00   & 0x00   & 0xAC \\ \hline
                \end{tabular}
            } 
        \end{table}

        \item[指纹删除] 在上位机中完成指纹删除操作,通过下行 UDP 包,删除指纹模块中对应 ID 的指纹特征信息。
        
        指纹删除功能主要在上位机中完成,在出现指纹出现问题的时候,由 HR 执行运行函数,自动化删除数据库中员工对应的指纹特征编号,
        同时将删除指纹特征编号的命令通过下面的指令(见表\ref{指纹特征清除})经由树莓派下发到指纹终端,根据不同的需求,在下面的帧中配置不同的格式。

        \begin{table}[H]
            \caption{指纹特征清除} \label{指纹特征清除}
            \resizebox{\textwidth}{!}{%
                \begin{tabular}{lllllll}
                \hline
                \multicolumn{1}{c}{校验密码} & CMD类型 & CMD号 & CL\_FLAG & ID\_H & ID\_L & 校验和  \\ \hline
                0x00 0x00 0x00 0x00        & 0x01  & 0x51 & 03       & 0xFF  & 0xFF  & 0xAC \\ \hline
                \end{tabular} 
            } 
        \end{table}

        \item[指纹考勤登记] 在下位机从属指纹模块中完成基于指纹的考勤实现。
        
        下位机会循环的向指纹模块询问指纹能否匹配(见表\ref{询问是否匹配}),并按照匹配应答包\ref{匹配应答包}格式返回对应结果,
        根据对应的匹配分数以及是否完成匹配。

        当当前指纹被实际判断匹配成功,对应的数据就会被放在下面的帧结构中,再通过 UdpSocket 发送到上位机的 Socket,
        上位机中利用 Python Socket 对于 5555 端口进行长期监听,对所有监听到的数据包进行匹配,当前后帧结构一致的情况下
        将其中包含的 ID 按照 sqlite 的规定插入到对应表中
        \footnote{sqlite中有一种名为 CURRENT\_TIMESTAMP 的默认属性,会默认将插入项的时间作为一个元素一起插入}。

        \begin{table}[H]
            \centering
            \caption{询问是否匹配,查询匹配结果} \label{询问是否匹配}
            \begin{tabular}{llll}
            \hline
            \multicolumn{1}{c}{校验密码} & CMD类型 & CMD号    & 校验和  \\ \hline
            0x00 0x00 0x00 0x00        & 0x01  & 0x21 / 0x22 & 0xAC \\ \hline
            \end{tabular}
        \end{table}

        \begin{table}[H]
            \caption{匹配应答包格式} \label{匹配应答包}
            \resizebox{\textwidth}{!}{%
                \begin{tabular}{llllllll}
                \hline
                \multicolumn{1}{c}{校验密码} & CMD类型 & CMD号 & 错误码                 & 匹配结果      & 匹配分数      & 匹配ID      & 校验和  \\ \hline
                0 0 0 0        & 0x01  & 0x51 & 0 0 0 0 & 0x00 0x01 & 0x27 0x0F & 0x00 0x03 & 0xAC \\ \hline
                \end{tabular} 
            }  
        \end{table}

        \item[考勤记录读取] 根据一般企事业单位的考勤系统的历史发展来看,读取员工考勤记录属于考勤系统的必备功能。
        
        本功能主要实现在上位机处,由上位机调用函数对于原先保存在 SQLite 中的考勤打卡表与员工-指纹特征对应表进行自动关联,并且
        基于一定的逻辑分析打印出员工的详细打卡数据与月度打卡数据。
        \newpage
    \end{description}
      

  % 第二章
% \section{系统开发流程}

    按照现有嵌入式企业的嵌入式软件开发流程,开发一个嵌入式系统主要分为以下几个步骤。

\subsection{用户需求分析}

    根据现有的各种企事业单位对于考勤打卡的需求,需求的特性主要为特征性识别以及日志记录。
    
    因此,在总体设计上我计划采用最简单的由指纹识别模块获取输入,经过 MCU 中简单处理再转发给
    linux 下的控制主机的设计。

\subsection{嵌入式开发环境的搭建与说明}

    由于嵌入式开发所基于的 MCU 一般性能相当有限,就算是采用一般 linux 操作系统进行本机编译,其占用时间也会相对比较长,同时,也无法应对一些占用系统内存资源较大的编译场景。
    因此,在嵌入式开发中,一般通过交叉编译的方式实现在 x86\_64-linux 平台或其他通用操作系统架构平台上实现对于目标平台代码的编译以更好的利用硬件资源。

    在本文的实现过程中基于 \href{https://nixos.org/}{Nix} 管理 x86\_64-linux 平台实现了对 aarch64-unknown-linux 目标平台的编译,其中主要的编译工具链的部分直接采用原有 \href{https://github.com/rcore-os/arceos}{ArceOS} 操作系统实现的 Rust 语言交叉编译以及镜像处理步骤,将 Cargo 包管理工具直接生成的裸机 elf 文件通过 rust-objcopy 去除其中一些无关的信息,如调试信息等内容成为一个纯粹的二进制文件。 
  
    一般树莓派的流程由 保存在 Soc ROM 区中的 booloader 完成 SD 卡上 FAT32 分区的挂载以及加载第二阶段 bootcode.bin,但是由于我使用的树莓派4B (bcm2711) 相对于前代有不少的硬件更新,在我使用的树莓派中对应初始化,启用 GPU fireware 加载 start.elf 的 bootloader 代码都被实现在 EEPROM 中,以提升 ROM 代码的容错性。 
    在运行 start4.elf 文件时,其会对 sd 卡中的 config.txt 文件进行解析,完成对应如串口传输频率,是否启用 JTAG 调试等配置,还会将其中声明的镜像文件加载到内核地址,使 CPU 由 stand-by 状态开始执行内核初始化代码。

    在我开发的过程中参照 \href{https://github.com/rust-embedded/rust-raspberrypi-OS-tutorials}{rust-raspberrypi-OS-tutorials} 的串口传输工具完成了串口传输的配置。其中通过实现一个最小配置内核,实现了初始化对应端口(PIN 14,15)的替代方法声明以启动对应端口的传输声明。同时通过CH304 USB 转 TTL 串口传输模块发送开始传输信号给开发机中 Ruby 运行的应用程序,应用程序将内核镜像文件通过串口传输到树莓派4B内存中以完成镜像加载。最终最小配置内核将控权转交给内核镜像文件。

    \subsubsection{基于 Nix 构建可重构开发环境}

    作为一个标准开发操作系统的开发环境,必然是需要在同组内保持一定程度上的可重构性,易重构性以及同步性。基于这几种考量,我选用了 \href{https://nixos.org/}{Nix} flakes 对于项目整体依赖进行管理。就目前来看,除了对于使用到其他项目中的 docker 的部分,由于在 Non-NixOS 中,Nix 无法介入 systemctl 的管理而存在一定的不一致情况以及由于 WSL 对于串口设备连接的限制\footnote{在WSL中连接串口设备的时候,需要额外安装 usbipd},其他的部分表现良好,均能很好的在 WSL, NixOS, Debian 等常用开发系统中构建一致,可用的开发环境。

    % 具体在实现过程中,我通过 flakes.inputs 固定了后面引用的 Nixpkgs, rust-overlays 库。
    % 同时,使用 overlays 在原先 nixpkgs 上掩盖了我自己的派生以保证开发环境构建的一致性。
    % 在代码段\ref{nix-flake-overlays} 第 7-10 行实现了对于 rust nightly toolchain 的固定,
    % 在后文 11-17 行实现了对于 nixpkgs 特定版本 qemu 的选择,在 18-21 行实现了对于联网获取的编译工具链的固定。
    
    \begin{lstlisting}[language=nix
        , caption=my flakes
        , label = {nix-flake-overlays}
        , numbers = left
        , breaklines=true
        , breakatwhitespace=true]
overlays = [ 
    (import rust-overlay)
    (self: super: {
        rust-toolchain =
        let rust = super.rust-bin; in
            # The rust-toolchain when i make this file, which maybe change
            (rust.nightly."2020-04-07".override {
            extensions = [ "rust-src" "llvm-tools-preview" "rustfmt" "clippy" ];
            targets = [ "x86_64-unknown-none" "riscv64gc-unknown-none-elf" "aarch64-unknown-none-softfloat" ];
            });
        qemu7 = self.callPackage "${nixpkgs-qemu7}/pkgs/applications/virtualization/qemu" {
        inherit (self.darwin.apple_sdk.frameworks) CoreServices Cocoa Hypervisor;
        inherit (self.darwin.stubs) rez setfile;
        inherit (self.darwin) sigtool;
        # Reduces the number of qemu source files from ~10000 to ~3619 source files.
        hostCpuTargets = ["riscv64-softmmu" "riscv32-softmmu" "x86_64-softmmu" "aarch64-softmmu" ];
        };
        x86_64-linux-musl-cross = fetchTarball {
        url = "https://musl.cc/x86_64-linux-musl-cross.tgz";
        sha256 = "172zrq1y4pbb2rpcw3swkvmi95bsqq1z6hfqvkyd9wrzv6rwm9jw";
        };
    })
    \end{lstlisting}

    同时,为了保证引入的工具链能完整的运行,我根据 \href{}{nixpkgs} 中提出的 issue,对于部分存在的问题进行了修复。

    \begin{lstlisting}[language=nix
        , caption=flakes 特殊适配
        , numbers = left
        , breaklines=true
        , breakatwhitespace=true]
unset OBJCOPY # Avoiding Overlay
export LIBCLANG_PATH="${pkgs.llvmPackages.libclang.lib}/lib" # nixpkgs@52447
export LD_LIBRARY_PATH="${pkgs.zlib}/lib:$LD_LIBRARY_PATH" # nixpkgs@92946

export PATH=$PATH:${pkgs.aarch64-linux-musl-cross}/bin: # ... etc
    \end{lstlisting}

%     \newpage

\subsection{ArceOS 操作系统现有驱动调用分析}

    下图\ref{fig::cvitek}左侧的部分是 ArceOS 操作系统的整体布局,右侧是现有\href{https://github.com/yuoo655/arceos_net/tree/hsp}{cvitek 物理网卡驱动} 的逐层调用情况。

    该 cvitek 物理网卡驱动主要作用在华山派,荔枝派等主机上。但与我们采用的树莓派4B中由 Soc 中集成 MAC 实现不一样,他们采用的这款设备提供了一个额外的以太网 MAC 控制器的 IP 核DWMAC 来完成 MAC 层的实现。

        
    \begin{figure}[ht]
        \centering
        \includegraphics[scale=0.4]{imgs/cvitek.jpg}
        \caption{cvitek 驱动调用栈}    \label{fig::cvitek}
    \end{figure}

    在整套实现中,ulib::libax 通过调用 module::axnet 所实现的底层方法(如TcpSocket,UdpSocket等)实现了用于 TCP/UDP 通信的通信源语。
    在 module::axnet 中这个部分的实现是针对于 
    \href{https://github.com/smoltcp-rs/smoltcp}{smoltcp} 这个 tcp/ip 协议栈进行了针对性的改造(非标准库环境等改造)而完成的。
    如果在编译的时候添加了对应的 features, ArceOS 会自动根据 module::axdriver::build.rs 文件中所进行的声明,
    将一个带有不同设备名称的 feature 加入默认 feature list 中,以方便实现基于设备组(phy, net, block, display) 实现的自动驱动加载。
    在完成 build.rs 编译脚本中的检查等操作之后,Cargo 在对于 axdriver 进行编译的时候,
    就会识别到当前编译携带了 cvitekphy/nic feature 从而根据 \#[cfg(feature = "cvitekphy")] 启用 cvitek\_traits.rs 的编译。
    cvitek\_traits.rs 文件将 ArceOS 上层 module 所提供给下层 crates 的方法支持(如dma\_alloc\_pages,delay等)
    通过 traits 的默认实现传递给下方的 crates 进行使用以实现 crates 层逆向调用 modules 层方法的效果。
    \footnote{在cvitek\_traits.rs 文件中 CvitekPhyTraits 声明并实现于 CvitekPhyTraitsImpl }。
    同时,在 module::axnet::driver.rs 中基于 cfg\_if 库的条件编译语句也实现了将 cvitek 网卡驱动转换成为 AxNetDevice 
    并实现了 Driver traits 下属的 probe\_global 方法的效果。
    最后在 module::axdriver 中 for\_each\_driver 宏的帮助下,ArceOS 将各个加载的网卡驱动转换成为 
    Driver 并运行(probe\_global)初始化,并将其添加到 AllDevices 下属的结构体中。

    而根据 ArceOS 的规定,cvitek 以太网卡驱动的实际实现实际上被封装在各自的 crates 中。在 crates::driver\_net 包规定了一系列网络设备、
    所必须要实现的 traits(如 transmit, receive)等方法。
    新的网络驱动会使用其内部方法实现这些对应的 traits,将这些方法包装成为 ArceOS 的调用方法。

\subsection{以太网卡驱动实现}

    根据前文的分析,如果想要实现在 TRANSPORT 层或者 NETWORK 层实现树莓派和主机之间的通信效果,主要需要实现以太网 OSI 七层模型中的 DATA\_LINK 与 PHYDIVSL 层之间的通信,
    由下图\ref{fig::dataLink}可知,主要需要实现的部分在于使 Soc 上的 MAC 实现能通过 GMII, RGMII,Serial-GMII 等接口标准与 PHY 芯片进行联通,进而调用 PHY 芯片上对于以太网传输介质上光,电等信号进行解析的方法。
        
    \begin{figure}[ht]
        \centering
        \includegraphics[scale=0.4]{imgs/data_link_layers.jpg}
        \caption{IEEE 802.8 数据链路层}    \label{fig::dataLink}
    \end{figure}

    基于 树莓派 4B Soc bcm2711 芯片手册,其板载 PHY 芯片 BCM54213PE 数据表以及 IEEE 802.3 协议手册可以了解到,目前 Soc 和 PHY 之间是存在直连的 RGMII 通信接口的。
    在 RGMII 通信接口中由 GTK\_CLK, RX\_CLK 实现双向时钟同步,TXD[0:3], RXD[0:3] 实现数据传输,TX\_CTL,RX\_CTL 实现数据传输控制,MDIO 和 MDC 实现 MAC 与 物理层的控制和状态关系的设置。
    \footnote{虽然在bcm2711数据手册GPIO替代函数表中有提到提供了 RGMII\_MDIO 与 RGMII\_MDC Pins,但是没有找到任何公开资料描述这个部分与底层之间是如何联系起来的}

    % TODO: 其中 GPIO 中也提供了一组 MDIO MDC 有点没太明白这段是为什么添加到这个地方?

    \subsubsection{概念介绍}
    % NOTE: 我个人感觉没有必要在这里介绍一系列 MDIO 芯片等是如何初始化的,但是如果要凑字数的话感觉也还行 (

    \subsubsection{源码分析}

    根据对于 uboot 源码所进行的分析,在以太网 PHY 芯片中维持了如下的数据结构\ref{fig::uboot-genet-struct}。

    \begin{minipage}[t]{0.42\linewidth}
    \begin{lstlisting}[columns=fullflexible, label={fig::uboot-genet-struct}]       +
0x0    | IOBASE
+------|-----------------------------+
0x2000 | GENET_RX_OFF
       | [DMA_DESC_SIZE;TOTAL_DESCS]
0x2C00 | GENET_RDMA_REG_OFF
       | [DMA_RING_SIZE;DEFAULT_Q]
0x3000 | RDMA_RING_REG_BASE
       | [DMA_RING_SIZE]
0x3040 | RDMA_REG_BASE
       | dma_reg
+------|-----------------------------+
0x4000 | GENET_TX_OFF
       | As above ...
       +
        \end{lstlisting}
    \end{minipage}
    \quad
    \begin{minipage}[t]{0.5\linewidth}
        \vspace{2em}
        \setlength{\parindent}{1em}
        其中,IOBASE 为 BCM54213PE 芯片基于 MMIO 地址映射,在树莓派内存中对应的地址。
        在树莓派网卡中,0x2000-0x4000的地址主要分配给与Rx相关的结构体。0x4000-0x6000的地址分配给TX相关的结构体。

        这些结构体都是以太网卡中 DMA 模块至关重要的成分。
        以 0x2000 GENET\_RX\_OFF 开头,到 GENET\_RDMA\_REG\_OFF 为止的一段地址中保存了 256 个 DMA 描述符结构。
    \end{minipage}

    以 0x3000 GENET\_RDMA\_REG\_OFF 开头,到 RDMA\_RING\_REG\_BASE 为止的这一段地址中保存了 BCM54213PE 
    所支持的 16 个不同优先级别的接受环(方便 DMA 实现基于不同优先级的接受),
    还在 RDMA\_RING\_REG\_BASE 与 RDMA\_RING\_REG\_BASE 之间保存了一个默认的接受环。

    也就是说在 BCM54213PE 的硬件实现中提供了对于 16 个优先级队列,以及一个默认队列的支持。不过根据源代码,
    实际上不管是树莓派官方的 linux 内核或者 uboot 下的以太网驱动都没有全部使用这些队列。
    在树莓派中,将 256 个 DMA 描述符分配给五个队列,其中前四个队列分别占有 32 个描述符,第五个,也就是默认队列
    占有剩余的 128 个描述符。而在 uboot 中的实现考虑到了其需求,对于原先的设计进行了删减,
    直接将 256 个 DMA 描述符全部分给了默认队列(并没有使用优先级队列)。
    
    \begin{adjustwidth}{0cm}{} % NOTE: 消除全局段首锁紧带来的影响
    \begin{minipage}[b]{0.5\linewidth}
        \setlength{\parindent}{2em} % 段首缩进
        在 BCM54213PE DMA 模块中以太网模块初始化的过程中,会逐个将使用的 Rings 环进行初始化,
        每一个抽象意味上的环都是由 start\_addr 到 end\_addr 中的一段连续的 DMA 描述符组合而成的。
        每一个DMA描述符指向一段特定长度的内存空间,这也就是后文所提到的缓冲区的概念\ref{code::InitRxRings}。

    \end{minipage}
    \hfill
    \begin{minipage}[b]{0.45\linewidth}
        \includegraphics[scale=0.6]{./imgs/Rings_and_Descs.jpg}        
        \captionof{figure}{DMA 描述符与环}
    \end{minipage}
    \end{adjustwidth}

    \newpage
    \begin{lstlisting}[language=C
        , caption={Initialize Rx priority queues}
        , label={code::InitRxRings}
        , numbers = left
        , breaklines=true
        , breakatwhitespace=true]
for (i = 0; i < priv->hw_params->rx_queues; i++) {
    ret = bcmgenet_init_rx_ring(priv, i,
                    priv->hw_params->rx_bds_per_q,
                    i * priv->hw_params->rx_bds_per_q,
                    (i + 1) *
                    priv->hw_params->rx_bds_per_q);
    if (ret)
        return ret;

    ring_cfg |= (1 << i);
    dma_ctrl |= (1 << (i + DMA_RING_BUF_EN_SHIFT));

static int bcmgenet_init_rx_ring(struct bcmgenet_priv *priv,
                unsigned int index, unsigned int size,
                unsigned int start_ptr, unsigned int end_ptr);
}
    \end{lstlisting}

    % \begin{figure}[htp]
    %     \flushleft
    %     \includegraphics[scale=0.4]{./imgs/Rings_and_Descs.jpg}        
    %     \caption{DMA 描述符与环}    
    %     % \label{fig::dataLink}
    % \end{figure}
    
    % % 也就是说在 BCM54213PE 的硬件实现中提供了对于 16 个优先级队列,以及一个默认队列的支持。不过根据源代码,
    % % 实际上不管是树莓派官方的 linux 内核或者 uboot 下的以太网驱动都没有全部使用这些队列。
    % % 在树莓派中,将 256 个 DMA 描述符分配给五个队列,其中前四个队列分别占有 32 个描述符,第五个,也就是默认队列
    % % 占有剩余的 128 个描述符。而在 uboot 中的实现考虑到了其需求,对于原先的设计进行了删减,
    % % 直接将 256 个 DMA 描述


    % % \begin{minipage}[t]{0.3\linewidth}
    % %     \includegraphics[scale=0.4]{./imgs/Rings_and_Descs.jpg}
    % %     \captionof{figure}{DMA 描述符与环}
    % % \end{minipage}
    % % \quad

    % % \begin{minipage}[t]{0.3\linewidth}
    % %     根据左侧的图以及 DMA 的相关说明,在 BCM54213PE 上 DMA 模块中,每一个环都相当于数据结构中的一个循环队列。
    % %     这个循环队列以
    % % \end{minipage}



  % 第三章
% \section{软件测试}

    根据本次毕业设计中的实现内容,软件测试主要分为两大部分,系统功能模块测试及系统整体测试,其中功能模块测试主要包含
    指纹识别模块测试,网卡驱动测试,网络协议栈集成测试几个部分。

    \subsection{指纹识别模块测试}

    本测试主要在上位机中进行,主要目的在于证明可以通过指纹管理程序在不影响指纹识别的情况下
    对于指纹特征数据进行上传和下载(见图\ref{总体设计图})右侧上位机与 FPM383C 之间的连线。

    \subsubsection*{测试准备:}
    硬件层面上需要使用 CH340 TTL转USB模块,杜邦线,指纹识别模块,开发板。
    软件层面上不需要额外准备,直接采用 chainboot 显示对应输出即可。

    经串口发送亮灯信号,对应指纹模块亮灯可见指纹识别模块能与计算机和树莓派建立有效的串口连接。

    \subsubsection{指纹注册测试}
    
    由于按照目前设计,指纹注册功能在上位机中实现。在正确注册指纹之后,通过调用协议使指纹模块上传
    指纹特征信息到上位机,通过 UdpSocket 将对应数据包中信息整合后经过网络分发到下位机
    \footnote{在这段测试中,由于经由网络下发指纹特征信息由于量太多不太容易进行呈现,暂且不进行相关演示}。

    首先通过指纹识别模块配的客户端程序清空上位机从属指纹模块中的指纹模板(如图\ref{fig::清空指纹模板})。

    \begin{figure}[ht]
        \centering
        \includegraphics[scale=0.8]{./imgs/清空指纹模板.png}
        \caption{清空指纹模板}    \label{fig::清空指纹模板}
    \end{figure}

    将指纹模块插入到上位机中,通过在上位机中运行的简单注册终端进行注册。
    首先,根据用户信息表(见表\ref{tab:userInfo})中所设计的各项信息进行输入,在输入数据完成校验。
    然后,管理程序调用指纹注册函数,函数在调用指纹模块的自动注册命令完成有关指纹模块的注册后
    自动调用上传指令,
    将对应数据由上位机从属指纹识别模块上传到上位机数据库中 Finger{id}Data 表中
    (见图\ref{test::用户注册} \footnote{图中下半部分输出的字节数据分别代表由串口获取的完整特征信息和其中的指纹信息}),
    并且根据数据库设计,自动完成 users,FingerInfo 等关联表的构建。

    \begin{figure}[ht]
        \centering
        \includegraphics[width=\textwidth]{./imgs/测试-用户注册.png}
        \caption{用户注册}    \label{test::用户注册} 
    \end{figure}   

    完成注册操作之后,由图\ref{test::注册后数据库查询} 可见,对应的表和关系已经正确构建。
    下图中对 FingerInfo 表的查询体现了 finger\_id 0 与 user\_id 1 的映射关系,还保存了指纹特征 0 的长度信息,
    注册的指纹特征信息二进制数据被保存在 Finger0Data 表中。

    \begin{figure}[ht]
        \centering
        \includegraphics[width=\textwidth]{./imgs/测试-注册后数据库查询.png}
        \caption{用户注册后数据库查询}    \label{test::注册后数据库查询}
    \end{figure}   

    \subsubsection{指纹删除}

    指纹删除功能终端尚未实现,还需要一些时间。

    \subsubsection{指纹下载}

    本测试不属于指纹考勤管理系统的主要组成成分,主要目的在于测试在指纹特征数据的上传和下载中是否出现了信息丢失
    导致指纹无法正常识别的现象。

    首先,先根据指纹注册测试的方式或者是由指纹模块配的指纹管理程序完成指纹注册(见图\ref{fig::注册指纹})后单独进行指纹上传操作。
    然后,由指纹模块配的指纹管理程序删除注册的指纹(见图\ref{fig::清空指纹模板})
    \footnote{删除测试的方法会一并删除指纹特征文件与关联图等内容}。

    \begin{figure}[ht]
        \centering
        \includegraphics[width=\textwidth]{./imgs/测试-指纹上传.png}
        \caption{指纹上传}    \label{test::指纹上传}
    \end{figure}   

    \noindent
    \begin{minipage}[t]{0.48\linewidth}
        \includegraphics[width=\textwidth]{./imgs/注册指纹.png}
        \captionof{figure}{指纹客户端注册指纹}    \label{fig::注册指纹}
    \end{minipage}
    % \quad
    \begin{minipage}[t]{0.48\linewidth}
        \includegraphics[width=\textwidth]{./imgs/测试-指纹上传后查询与匹配.png}
        \captionof{figure}{指纹上传后查询与匹配}    \label{test::指纹上传后查询与匹配}
    \end{minipage}

    如图 \ref{test::指纹上传} 可见通过执行内置指纹上传函数实现的指纹上传输出,
    其中 (10, [ ..., 0, 81, ... ]) 代表执行 upload 指令时的回返信息,
    第一段数组代表由数据库中读取的二进制指纹特征信息(一帧),
    第二段字节代表整个由上位机经由串口下发到指纹识别模块的串口信息。

    将指纹识别模块重新接入上位机中(如图\ref{test::指纹上传后查询与匹配})可见,
    之前删除的左手大拇指指纹已经重新被识别出来了,并且经过两次指纹匹配测试均能保证
    识别准确率能够达到 100,这无疑证明了在指纹特征信息由指纹模块中上传再下载的过程中
    并没有对于指纹特征信息的完整性进行破坏,上传再下载的行为也不会导致指纹特征信息被破坏。

    \subsection{以太网卡驱动测试}

    在本章节中主要完成了对于网卡驱动的测试工作,其中首先在两台 PC 电脑之间运行 C 语言的收发包脚本,对于脚本
    的功能以及效果进行测试,然后在树莓派原生 Linux 下运行发包脚本,PC 电脑上运行收报脚本以对于树莓派的传输速率进行分析,
    最后,在 ArceOS 上直接发送以太网帧到 PC 机上的接受脚本中测试 ArceOS 网卡驱动的收发包效果。

    \subsubsection{测试脚本测试}

    按照测试计划,采用两台电脑先对测试脚本本身能否完整实现功能进行测试,一台电脑通过 WSL(windows subsystem of linux) 
    运行 linux 测试发包脚本,另外一台笔记本通过 RJ45转USB模块\footnote{或者说叫做usb网卡}
    经由双绞线与前者进行连接,接受对应数据包。

    但是进行测试的时候,实际上并没有办法实现以太网帧的收发,
    经过分析之后发现产生这种问题的主要原因的是 WSL 与 宿主机 之间的网络连接方式是通过
    类似于 NAT(Network address translation) 的方式进行连接的,在这种情况下,WSL 与 宿主机之间存在一个子网,因此
    在 WSL 中在以太网帧层面实现的收发只能在,并不能在宿主机上获得转发,因此
    另外一台 linux PC 没有办法收到对应以太网帧
    \footnote{IP层以上的包会自动被转发,但以太网帧不会}。

    最终,采用了两台运行 linux 操作系统的电脑进行此测试(其中一台采用 nixos live,一台是 nixos)其连接如下图 \ref{tests::测试脚本测试连接图}所见。
    \footnote{由于其中一台仍通过 RJ45转USB模块 实现网络收发功能同时配有百兆网卡,因此本次测试中的最高传输速率仅为100mbps}
    二者均在完全关闭防火墙,配置对应以太网卡的 ip 地址,并实现相互 ping 之后展开测试。
    发包侧(如附录代码段 \ref{code::eth-send})在十秒内持续发送以太网帧,并且在以太网帧的第 16 位开始的八个字节添加代表当前以太网帧序号的八个字节 (unsigned long long)数据。
    收包侧(如附录代码段 \ref{code::eth-recv})按照对应收包顺序与检测出的序号综合进行判断在传输过程中是否存在丢包,乱序的情况。

    \begin{figure}[ht]
        \centering
        \includegraphics[scale=0.6]{imgs/测试脚本测试连接图.jpg}
        \caption{测试脚本测试连接图}    \label{tests::测试脚本测试连接图}
    \end{figure}

    具体测试结果见下 4-1 \ref{test::测试脚本测试}表,在以太网帧百兆层面不存在丢包的情况,可以较好的实现收发包测试效果。

    \begin{table}[ht]
    \centering
    \label{tests::测试脚本测试}
    \caption{测试脚本测试表}
    \csvautobooktabular{./imgs/测试记录-测试脚本测试.csv}
    \end{table}

    \subsubsection{树莓派网络传输速率测试}

    本测试主要想要实现的效果在于测试出在树莓派直接采用 Linux 镜像原装的网卡驱动所能实现的传输效率,以方便与后面
    我们通过 Rust 在 ArceOS 裸机上实现的网卡驱动(以太网帧)测试出的收发效率进行对比。

    在具体测试操作上,除了其中一台百兆网卡的测试PC被更换成为了树莓派外基本与前章节(测试脚本测试)保持一致。
    
    \begin{figure}[ht]
        \centering
        \includegraphics[scale=0.6]{imgs/树莓派-测试机连接.jpg}
        \caption{树莓派-测试机连接图}    \label{tests::树莓派测试机连接图}
    \end{figure}


    此处采用以太网帧作为其中一个测试参数主要是考量到在测试的过程中采用死循环不断发送的方法,
    单个以太网帧长度越长,在发送该以太网帧花费的传输时间就越长
    \footnote{由于拷贝,寄存器操作等导致},
    花费在帧与帧之间切换的时间就越短,更能测试出以太网驱动的极限网络吞吐量。

    在进行测试的时候采用如下算法对于以太网传输速率进行评估,其中 total\_bytes 代表在测试时间内完成传输的总字节数,elapsed\_secs 代表总测试时间。
    经过测试,(测试记录见附录表 \ref{tests::Linux驱动传输速率测试}),将每组数据取平均值之后可做出如下图像(如图 \ref{tests::Linux网卡驱动传输率与以太网帧大小} 所示)

    由图可知,前文提出的以太网帧与网络传输速率之间的关系基本正确,二者存在指数关系,且以太网帧越大,在同等时间内的以太网传输速率越大。
    在以太网帧层面网络传输速率最大能达到 983.21 MBps\footnote{以太网帧总长度 1500 字节情况下},接近树莓派千兆网卡的理论最大传输速率 1000 MBps。

    $$\text{mbps} = \frac{\text{total\_bytes} \times 8.0}{\text{elapsed\_secs} \times 1,000,000.0}$$

    \begin{figure}[ht]
    \begin{tikzpicture}
        \centering
        \begin{axis}[
            xlabel={以太网帧大小}, ylabel={传输速率},
            xmin=0,
            width=0.8\textwidth,
            legend pos=south east,
            grid=both
        ]
        
        \addplot[thick, mark=*, blue] coordinates {
            (64, 112.3833333)
            (128, 233.0333333)
            (256, 456.5257143)
            (512, 739.01)
            (1024, 898.616)
            (1280, 943.565)
            (1408, 949.85)
            (1500, 946.8366667)
        };
        \addlegendentry{发送端速率}
        
        \addplot[thick, mark=*, red] coordinates {
            (64, 112.3833333)
            (128, 233.0333333)
            (256, 445.5085714)
            (512, 734.5766667)
            (1024, 889.338)
            (1280, 938.1825)
            (1408, 949.85)
            (1500, 938.5266667)
        };
        \addlegendentry{接受端速率}

        \end{axis}
    \end{tikzpicture}
    \caption{树莓派Linux驱动传输率与以太网帧大小关系图}    \label{tests::Linux网卡驱动传输率与以太网帧大小}
    \end{figure}

    % TODO: 在前期测试中是可以正常发现对应丢包情况的发生的,但是不知道为什么后面无法重现这个问题,暂且删除这个部分的说明

    % 在树莓派进行测试的过程中,根据序号与收包端收报次序的对比,C 接收端程序提示在本次测试中出现了丢包的情况。
    % 经过 Wireshark 对比之后证实确实存在丢包情况。

    % TODO: 在拷贝图片的时候两张图片是一致的li
    % \begin{figure}[ht]
    %     \centering
    %     \includegraphics[width=\textwidth]{./imgs/丢包2.png}
    %     \includegraphics[width=\textwidth]{./imgs/丢包1.png}
    %     \caption{以太网帧发包测试}    \label{test::以太网帧发送}
    % \end{figure}   

    \subsection{网卡驱动测试}

    本测试主要目的在于测试在 ArceOS 下的以太网卡驱动传输速率,并且最终与树莓派网络传输速率进行对比,评价以太网网卡驱动的实现效果。
    由于后面在将以太网卡驱动与 ArceOS 网络协议栈进行嵌合的时候对于以太网卡驱动进行了破坏性修改
    \footnote{BCM54213PE驱动中调用ArceOS函数是通过PhantomData间接调用traits实现来完成的}。
    因此,本次测试由完成了网络驱动测试的 \href{https://bitbucket.org/jackyliu16/arceos/commits/92e9b6abcdf180359381088552688c0fcbc83bf2}{commit} 
    分叉出 \href{https://bitbucket.org/jackyliu16/arceos/commits/branch/ethernet-test}{ethernet-test}
    分支,并在此分支中完成了对应测试。

    \subsubsection{发包测试}

    由树莓派向上位机直接发送以太网帧,在上位机中通过 wireshark 进行抓包(如图\ref{test::以太网帧发送})

    \begin{figure}[ht]
        \centering
        \includegraphics[width=\textwidth]{./imgs/以太网帧通信正常.jpg}
        \caption{以太网帧发包测试}    \label{test::以太网帧发送}
    \end{figure}   

    \subsubsection{收包测试}

    在树莓派上直接收包并打印对应信息(见图\ref{test::以太网帧接收})。
    由于目前树莓派并没有实现网络协议栈,因此
    无法通过如 nc, nping 等基于网络的传输工具向树莓派发送自定义以太网帧。
    因此,此处直接使用了 nc 随便发送了一段数据到树莓派中。
    此处指定的网段与上,下位机网段一致,因此上位机会在同一子网中释放 ARP 探寻包,
    如图 \ref{test::以太网帧接收} 所示,对应的 ARP 探寻包被正确解析,因此可以浅薄的认为以太网驱动可以正常的收到上位机传输的探寻包。
    % TODO: 这个地方其实可以补测试,用前面测试脚本测试的 send_ethernet.c 来发

    \begin{figure}[ht]
        \centering
        \begin{subfigure}
            \centering
            \includegraphics[width=\textwidth]{./imgs/测试-以太网帧接收.png}
        \end{subfigure}
        \hfill
        \begin{subfigure}
            \centering
            \includegraphics[width=\textwidth]{./imgs/测试-以太网帧接收-解析.png}
        \end{subfigure}
        \caption{接受的以太网帧及解析} \label{test::以太网帧接收}
    \end{figure}

    \subsubsection{传输速率测试}

    前面的测试说明当前以太网帧的收发功能是正常的,此处使用一个简单的发送 buf 的循环尝试对于以太网驱动的传输速率进行测试,
    其实现上类似于前文进行 Linux 原生网卡驱动测试时的操作,直接在以太网帧 16-24 位中插入代表当前序号的纯数字
    \footnote{由于测试脚本测试实现的部分为了方便进行调试,找到中断包,因此直接采用将代表序号的数字经由 snprintf 转换成 char 直接写入到以太网帧中
    ,此处为了保证一致此处采用了同样的方法}。

    $$\text{mbps} = \frac{\text{total\_bytes} \times 8.0}{\text{elapsed\_secs} \times 1,000,000.0}$$

    根据测试记录(见附录表 2-2),其中以太网帧大小与以太网帧传输速率之间同样存在对数关系(如图\ref{tests::ArceOS网卡驱动传输率与以太网帧大小}所示)。

    \begin{figure}[H]
    \begin{tikzpicture}
        \centering
        \begin{axis}[
            xlabel={以太网帧大小}, ylabel={传输速率},
            xmin=0,
            width=0.8\textwidth,
            legend pos=south east,
            grid=both
        ]
        
        \addplot[thick, mark=*, blue] coordinates {
            (60, 195.29)
            (128, 409.73)
            (256, 695.9225)
            (384, 941.24)
            (512, 955.32)
            (768, 969.84)
            (1024, 977.3)
            (1280, 981.84)
            (1408, 983.52)
            (1500, 984.55)
        };
        \addlegendentry{发送速率}
        
        \addplot[thick, mark=*, red] coordinates {
            (60, 189.773072)
            (128, 365.4372917)
            (256, 685.4498305)
            (384, 937.4238723)
            (512, 950.7467263)
            (768, 966.6822143)
            (1024, 973.7869313)
            (1280, 974.960128)
            (1408, 975.087309)
            (1500, 979.688400)
        };
        \addlegendentry{接受速率}

        \end{axis}
    \end{tikzpicture}
    \caption{树莓派ArceOS驱动传输率与以太网帧大小}    \label{tests::ArceOS网卡驱动传输率与以太网帧大小}
    \end{figure}

    经由二者横向简单对比,ArceOS 驱动在同等以太网帧大小情况下相较于 Linux 原生网卡驱动而言传输速率提升了不少(见 \ref{tests::ArceOSLinux传输速率对比图}),
    但由于 ArceOS 网卡驱动实际上不对发包过程进行任何形式的保证,因此其丢包率较 Linux 原生网卡驱动显著提高了不少。
    \footnote{但是由于数据链路层实际上不保证一定能发包成功,重传等工作实际上由以太网协议栈完成,因此相对来说影响面不大。}
    同时,在以太网帧相对较大的时候,发包端的成功包数量计算不正确,怀疑存在没有被处理的发包报错信息导致该包没有实际被发送,
    而是在 can\_transmit 层面判定当前缓冲区能否接受新的包时返回,但返回结果不正确导致正常发包与溢出并没有被明确区分开来,最终导致发包端发包结果记录显著高于
    收包端收到的数据包数量(见附录表 2-2 \ref{tests::ArceOS驱动传输速率测试} )
    
    \begin{figure}[H]
    \begin{tikzpicture}
        \centering
        \begin{axis}[
            xlabel={以太网帧大小}, ylabel={传输速率},
            xmin=0,
            width=0.8\textwidth,
            legend pos=south east,
            grid=both
        ]
        
        \addplot[name path=linux, thick, mark=*, orange] coordinates {
            (64, 112.3833333)
            (128, 233.0333333)
            (256, 445.5085714)
            (512, 734.5766667)
            (1024, 889.338)
            (1280, 938.1825)
            (1408, 949.85)
            (1500, 938.5266667)
        };
        \addlegendentry{Linux 驱动}
        
        \addplot[name path=arceos, thick, mark=*, blue] coordinates {
            (60, 189.773072)
            (128, 365.4372917)
            (256, 685.4498305)
            (384, 937.4238723)
            (512, 950.7467263)
            (768, 966.6822143)
            (1024, 973.7869313)
            (1280, 974.960128)
            (1408, 975.087309)
            (1500, 979.688400)
        };
        \addlegendentry{ArceOS 驱动}

        \addplot fill between[
            of = linux and arceos,
            soft clip={domain=0:1500},
            every even segment/.style  = {gray,opacity=.4}
        ];

        \end{axis}
    \end{tikzpicture}
    \caption{ArceOS-Linux驱动传输速率对比图}    \label{tests::ArceOSLinux传输速率对比图}
    \end{figure}

    \subsection{网卡协议栈集成测试}

    通过双绞线将树莓派的 RJ45 网口与笔记本端口相连,在笔记本上运行 server.py 程序(打开 5555 端口)
    清空所有防火墙设置,并且设置端口 5555 的出入站规则,将笔记本以太网网卡 ip 地址设置为
    树莓派默认 ip 地址 10.0.2.15 同一网段下的 10.0.2.16\footnote{见指纹识别模块flakes初始化语句}。

    \subsubsection{Udp 发包测试}

    在应用程序中,使用 axstd 替代 rust std 库,创建 UdpSocket,调用 sendto 方法,
    向上位机的 5555 端口发送测试连接字节数组(如图\ref{test::Udp发包}所示)。

    \begin{figure}[ht]
        \centering
        \includegraphics[width=\textwidth]{./imgs/测试-udp发包正常.png}
        \caption{Udp发包正常}    \label{test::Udp发包}
    \end{figure}   

    \subsubsection{Udp 收包测试}

    UdpSocket 收包测试我主要根据两种常见的网络联通方式进行测试,即 Netcat 和 ping。
    首先,针对 ARP,ICMP 包的测试(如图\ref{test::pingICMP回返测试}所示)。
    其中左上角是串口面板,右上角是正在运行中的 server.py 程序,左下角是
    执行的指令,即 ping 10.0.2.15。
    
    根据 ping 的一般操作流程,首先上位机中会先通过 ARP 探寻包,询问 10.0.2.15
    IP 地址所对应的硬件 MAC 地址,在下位机回返 MAC 地址后,连接被建立。
    上位机发送 ICMP Echo 请求到下位机,下位机以太网协议栈对收到的 ICMP 包进行处理,
    发送 ICMP Echo 作为回复。

    \begin{figure}[ht]
        \centering
        \includegraphics[width=\textwidth]{./imgs/测试-pingICMP回返测试.png}
        \caption{ping 回返测试}    \label{test::pingICMP回返测试}
    \end{figure}   

    然后,根据指纹考勤系统由上位机经由网络向下位机发送指纹信息的需求,
    我对于 Udp 连接进行了简单的测试。即通过 netcat 直接将一个字节数组
    (指纹模块的 LED 控制信号)
    由上位机传输到下位机,由图\ref{test::接收netcat数据}所示,下位机中
    缓冲区的字节数据与上位机中通过 netcat 传输的字节数据完全一致
    \footnote{这个地方没有对边界进行判断,直接打印了整个缓冲区}。

    \begin{figure}[H]
        \centering
        \includegraphics[width=\textwidth]{./imgs/测试-接收netcat数据.png}
        \caption{接收netcat数据}    \label{test::接收netcat数据}
    \end{figure}   

    \subsubsection{传输速率测试}
    
    在进行 UDP 层面传输速率测试的时候,在物理层面上采用类似前章节中对以太网帧进行测试时的连接结构。
    同时,软件层面上额外在上位机 10.0.2.16 中运行打开特定端口的程序避免出现报错返回包,直接通过 wireshark 对于收到的数据包
    数量进行记录(测试记录见附录表 2-3 \ref{tests::ArceOS-UDP传输速率测试})。
    
    其中采用的数据传输速率计算方法如下,其中 total\_bytes = UDP数据包数据部分长度 * 收发包数量,将收发包结果与前文中以太网帧
    层面的测试结果进行对比(如图 \ref{tests::ArceOSUDPETH对比图} 所示),可以明显发现 以太网帧传输速率 与 UDP帧传输速率存在较大差距,
    可以明显发现二者之间存在较大差距。

    根据初步的分析可能有两种原因导致这样的结果,由于 axnet, BCM54213PE 驱动兼容层的实现不正确损失了性能,或者是在 axnet 基于
    smoltcp 实现的网络协议栈中由于实现或者重复发包导致了性能损失。
    
    $$\text{mbps} = \frac{\text{total\_bytes} \times 8.0}{\text{elapsed\_secs} \times 1,000,000.0}$$

    \begin{figure}[H]
    \begin{tikzpicture}
        \centering
        \begin{axis}[
            xlabel={UDP 数据位长度}, ylabel={传输速率},
            xmin=0,
            width=0.8\textwidth,
            legend pos=north west,
            grid=both
        ]
        
        \addplot[name path=linux, thick, mark=*, orange] coordinates {
            (60, 189.773072)
            (128, 365.4372917)
            (256, 685.4498305)
            (384, 937.4238723)
            (512, 950.7467263)
            (768, 966.6822143)
            (1024, 973.7869313)
            (1280, 974.960128)
            (1408, 975.087309)
            (1500, 979.688400)
        };
        \addlegendentry{Linux 驱动}
        
        \addplot[name path=arceos, thick, mark=*, blue] coordinates {
            (128, 7.163255467)
            (256, 11.7116928)
            (384, 19.4936832)
            (512, 28.20819627)
            (640, 35.10186667)
            (768, 37.184256)
            (896, 46.4930816)
            (1024, 54.84489387)
            (1280, 65.50016)
            (1408, 72.410624)
            (1500, 0)
        };
        \addlegendentry{UDP 传输速率}

        \addplot fill between[
            of = linux and arceos,
            soft clip={domain=0:1500},
            every even segment/.style  = {gray,opacity=.4}
        ];

        \end{axis}
    \end{tikzpicture}
    \caption{ArceOS-UDP Eth对比图} \label{tests::ArceOSUDPETH对比图}
    \end{figure}
    
    \subsection{集成测试}

    集成测试部分主要测试了整个指纹考勤系统的集成功能,根据前文所提到的总体设计图,分为两个部分进行测试(如图 \ref{集成测试示意图} 所示)
    其中黄色部分被涵盖在指纹打卡及信息上传部分,红色部分被涵盖在指纹同步部分。

    \begin{figure}[ht]
        \centering
        \includegraphics[scale=0.6]{./imgs/总体设计图-plus.png}
        \caption{集成测试示意图} \label{集成测试示意图}
    \end{figure}

    \subsubsection*{测试准备:}

    按照前文中测试脚本测试时使用的外部网线连接方式(如下图 \ref{集成测试连接图} 所示),将树莓派与测试主机连接到一起。
    在测试主机中运行 server.py 应用程序,该应用程序会打开对于当前主机 5555 端口的监听,并且自动分析在当前端口下的数据通信。


    根据设定,该端口会自动分析应用程序发来的 UDP 包究竟是用于测试连通性的 TEST\_CONNECTION 还是下位机汇报某指纹 ID 实现打卡成功的 PUNCH\_RECEIVED 包,
    根据其类型做出不同的处理,并且返回对应的信息。

    \begin{figure}[ht]
        \centering
        \includegraphics{./imgs/集成测试连接图.jpg}
        \caption{集成测试连接图} \label{集成测试连接图}
    \end{figure}

    \subsubsection{指纹打卡及信息上传}

    根据设计要求,在一般情况下,树莓派中不断轮询指纹识别模块,在指纹识别模块收到正确匹配信号之后,提取对应 UART帧 中保存的指纹 ID 数据,将此数据写入到 PUNCH\_RECEIVED 的
    ASCII 码中,直接通过 UDP 发送裸字节包到上位机中运行的 server.py 应用程序,在该应用程序中自动对 PUNCH\_RECEIVED 前缀进行识别,识别成功之后查询数据库,给该指纹 ID
    所对应的用户添加打卡记录。

    \begin{figure}[ht]
        \centering
        \includegraphics[width=\textwidth]{./imgs/集成测试-打卡信息上传.png}
        \caption{打卡信息上传示意图} \label{打卡信息上传示意图}
    \end{figure}

    如图 \ref{打卡信息上传示意图} 所示,左侧运行的 server.py 应用程序在对于当前网络下的 UDP 
    传输进行监听的时候,收到了两种不同的数据包 TEST\_CONNECTION 和 PUNCH\_RECEIVED
    当收到下位机打卡信息汇报之后,根据 FingerInfo 表查询指纹 ID 对应的用户名称,并将对应用户的打卡
    信息写入到 signIn 数据库。如图 \ref{打卡信息上传示意图} 右侧所示,
    在 signIn 数据库中以 CURRENT\_TIMESTAMP 的方式保存对应的打卡信息,
    打卡时间以上位机当前时间戳对应的格林尼治时间进行显示(在进行测试的时候上位机位于东八区)。

    \subsubsection{指纹同步}

    指纹同步测试在 \href{https://bitbucket.org/jackyliu16/arceos/commits/4554a70c7b902b3fcae488e9507377aff42fedd9}{arceos-4554a70} 提交中展开,
    其中使用的脚本在 \href{https://bitbucket.org/jackyliu16/attendance_system_backend/commits/921212f1f126da11868c32f7cba1502972fd6576}{backend-921212f}。
    在测试开始前清空树莓派从属指纹识别模块中的指纹模板(如图 \ref{tests::清空指纹模板})。
    在测试的时候,在上位机中打开 server.py 避免 5555 端口未被监听的报错,在树莓派商店,串口载入对应镜像完成之后,运行 mylib.py 中的同步测试命令,该命令会将
    当前数据库中保存的指纹特征数据下发到树莓派从属指纹识别模块中。

    \begin{figure}[ht]
        \centering
        \begin{subfigure}
            \centering
            \includegraphics[width=\textwidth]{./imgs/清空指纹模板.png}
        \end{subfigure}
        \caption{清空指纹模板} \label{tests::清空指纹模板}
        \hfill
        \begin{subfigure}
            \centering
            \includegraphics[width=\textwidth]{./imgs/注册指纹.png}
        \end{subfigure}
        \caption{指纹下载成功} \label{tests::指纹下载成功}
    \end{figure}

    根据当前胶水层(数据链路层与ArceOS 上层网络协议栈嵌合层)实现情况,目前尚没有办法能较好的处理将 UDP 存放到缓冲区中的问题,因此在指纹同步测试部分采用了一个相对较为简易的解决办法,
    即通过给两端添加延时来实现 UDP 收发包双端的同步。

    根据树莓派下位机中运行的嵌入式设备主循环 (见代码段 \ref{algorithm::fingerprint_network_comm}),在 local\_socket.recv\_from(\&mut buf) 从 ArceOS 上层网络协议栈中获取到第一个 UDP 包之后,
    应当进入一个持续的接受状态
    \footnote{根据 FPM383C 通信协议,下发指纹特征信息时,需要先发送指纹特征信息下载包,向指纹识别模块说明下载情况,然后上位机需要连续向下位机中发送多个固定长度(128)的指纹特征信息包},
    因此,在判定进入接收状态之后,主循环自动跳过指纹匹配\footnote{指纹下载过程不能被指纹匹配命令打断}后,继续接受并转发下一个 UDP 包,直到完成全部包的接受。

    在实际进行测试的时,发现现有串口传输驱动虽然可以实现简单的发送,解析与接受效果,但在面临连续发送较长字节时仍存在溢出问题。
    具体表现在传输常见 LED 控制信号(长度 22 字节),或者指纹特征信息下载包(长度 23 字节)时可以正常发送,但是在传输指纹特征数据下载包(长度 148 字节)时会出现由于部分字节没有正确传输,
    使指纹识别模块误认为传输还没有终止的死锁状态。
    
    交叉对比同一抽象类 SenderInterface 下的 SerialSender,UdpSocketSender 实现的效果及输出结果可以认为二者传输的数据是等价的,考虑到
    在实现串口传输驱动的时候采用的是 FIFO 方式实现的对应操作,在实际发送数据的时候可能因没能及时清空缓冲区导致溢出丢失字节进而导致死锁,手动在每次发送字节时添加了 50\_0000 纳秒的延迟后
    能正常传输较大字节数据。在包级别比对之后,发现在正常传输数据时,在树莓派端 recv\_from 处所接收到的包存在丢包情况。
    但是在启用 socket blocking 模式下无法复现该情况,初步认为该情况可能与尚未完整实现的胶水层有关(RxToken,TxToken中牵涉到 recycle 的问题),同时由于目前尚未实现多线程,在主循环上使用阻塞收报
    方式会导致死锁问题而无法避免使用 nonblocking socket。最终在 UDP 发包端每次发包间隔 0.092s,每次串口发送字节间隔 50\_0000 纳秒的情况下实现了最为基础的指纹同步功能。

    最终,在 mylib 测试脚本中携带的指纹模板总数探测包返回当前存在一个指纹模板,同时将 FPM383C 指纹模板切换到 PC 机中,对应的指纹模板也能够正常被检出(如图 \ref{tests::指纹下载成功})。
    



% \section{结论}

本论文提供的成果作为 ArceOS 组件化操作系统驱动程序计划提交到主仓库中,预计在该操作系统树莓派支持上占据一定地位。

本论文以一个简单的嵌入式入门项目,考勤系统为例,在尚未实现对应驱动的 ArceOS 上实现其所依赖的必要驱动如 
FPM383C 指纹识别模块驱动,BCM54213PE 网卡通信驱动等,论证了组件化操作系统在一般嵌入式开发场景中的作用。

在本次论文的开发中,我完成了基于 ArceOS 的开发环境搭建如交叉编译链构建,Rust语言工具链配置,基于nix的开发环境搭建等。
重点分析了如 u-boot, circle, 树莓派 linux 操作系统在 树莓派 4B 上的 BCM54213PE PHY 网络驱动,
了解了驱动实现过程中牵涉到的 MDIO 通信协议,DMA(直接记忆存取)机制。
基于前人的工作将其深度嵌合到 ArceOS 操作系统中,利用现有 ArceOS 操作系统提供的上层以太网协议栈支持,
并最终完成了 ArceOS 操作系统的 udpsocket 支持\footnote{目前仅完成了这个部分的测试}。
同时,基于 FPM383C 指纹识别模块通信协议,在树莓派现有基于寄存器实现的 UART 支持下,通过对于寄存器进行操作
实现了基于串口的指纹识别模块通信解析与发送。

最终,在应用层面实现了一个能完成基本指纹考勤打卡功能的嵌入式系统。其中包括各个打卡终端,定时进行指纹采集,
将采集结果通过网络发送给上位机,上位机收到对应数据信息之后将打卡记录记录到数据库中以便查阅\footnote{计划还完成
上位机在收到对应信息之后的反馈,但目前由于收包部分还有部分合并工作没有完成,尚处于停滞状态}。
同时,还实现了指纹注册并下发机制,该机制由上位机实现,在上位机从属指纹识别模块录入指纹之后,上位机会基于一个 python 实现的
串口通信机制,获取对应从属模块的数据并通过网络下发到所有客户端。
\footnote{计划中完成定期指纹模板检测,自动发送对应特征长度给上位机进行检测,覆盖不匹配的数据}

虽然本文的工作取得了初步的成功,完成了树莓派 4B 上的基础网络支持,但要使 ArceOS 组件化操作系统的功能更加强大,更加完善,还需要很多工作要做。
同时,本文仅仅在一个相对比较完善的嵌入式平台(树莓派)上完成了驱动的改写,
虽然利用到了组件化操作系统所提供的条件编译带来的操作系统镜像大小优势,但是并没有尝试利用这一优点
\footnote{目前利用到这个优势的地方仅有通过串口传输二进制镜像文件,
ArceOS 编译出的镜像文件有且仅有不到200kb,相较于一般嵌入式操作系统大小小很多}实现什么特别的操作,
也没有通过降级 MCU 体现 ArceOS 操作系统可能的成本优势,同时 BCM54213PE PHY 网络芯片是一个专有的网络芯片,
基本上在树莓派以外的设备中很少见到有嵌入式设备使用了这款芯片,选择的驱动普世性不强。
只是我的实现,论证了确实 ArceOS 系统是可以
用于嵌入式开发的,在开发难度上相对 Exokernl 等较低,但驱动重写实际上工作量不低,可能对于现有设备
厂商而言,开发成本不太经济。


% 参考文献数据库(所有需要引用的参考文献写入references.bib文件中)
% \begin{references}
%     \bibliography{references.bib}
% \end{references}

% \appendix
% \nonumsection{附录 - 部分代码}

\begin{lstlisting}[language=nix
    , caption={可重建编译环境配置文件}
    , label = {nix-flake}
    , numbers = left
    , breaklines=true
    , breakatwhitespace=true]
{
description = "ArceOS Development Environment";

inputs = {
  nixpkgs.url      = "github:NixOS/nixpkgs/nixos-unstable";
  nixpkgs-qemu7.url = "https://github.com/NixOS/nixpkgs/archive/7cf5ccf1cdb2ba5f0
  8f0ac29fc3d04b0b59a07e4.tar.gz";
  rust-overlay.url = "github:oxalica/rust-overlay";
  flake-utils.url  = "github:numtide/flake-utils";
};

outputs = { self, nixpkgs, nixpkgs-qemu7, rust-overlay, flake-utils, ... }:
  flake-utils.lib.eachDefaultSystem (system:
  let
    overlays = [ 
    (import rust-overlay)
    (self: super: {
      # ref: https://github.com/the-nix-way/dev-templates
      rust-toolchain =
      let
          rust = super.rust-bin;
      in
      if builtins.pathExists ./rust-toolchain.toml then
        rust.fromRustupToolchainFile ./rust-toolchain.toml
      else if builtins.pathExists ./rust-toolchain then
        rust.fromRustupToolchainFile ./rust-toolchain
      else
        # The rust-toolchain when i make this file, which maybe change
        (rust.nightly.latest.override {
          extensions = [ "rust-src" "llvm-tools-preview" "rustfmt" "clippy" ];
          targets = [ "x86_64-unknown-none" "riscv64gc-unknown-none-elf" "aarch64-unknown-none-softfloat" ];
        });
      qemu7 = self.callPackage "${nixpkgs-qemu7}/pkgs/applications/virtualization/qemu" {
        inherit (self.darwin.apple_sdk.frameworks) CoreServices Cocoa Hypervisor;
        inherit (self.darwin.stubs) rez setfile;
        inherit (self.darwin) sigtool;
        # Reduces the number of qemu source files from ~10000 to ~3619 source files.
        hostCpuTargets = ["riscv64-softmmu" "riscv32-softmmu" "x86_64-softmmu" "aarch64-softmmu" ];
      };
      x86_64-linux-musl-cross = fetchTarball {
        url = "https://musl.cc/x86_64-linux-musl-cross.tgz";
        sha256 = "172zrq1y4pbb2rpcw3swkvmi95bsqq1z6hfqvkyd9wrzv6rwm9jw";
      };
      aarch64-linux-musl-cross = fetchTarball {
        url = "https://musl.cc/aarch64-linux-musl-cross.tgz";
        sha256 = "05cwryhr88sjmwykha5xvfy4vcrvwaz92r9an7n5bsyzlwwk0wpn";
      };
      riscv64-linux-musl-cross = fetchTarball {
        url = "https://musl.cc/riscv64-linux-musl-cross.tgz";
        sha256 = "119y1y3jwpa52jym3mxr9c2by5wjb4pr6afzvkq7s0dp75m5lzvb";
      };
    })
    ];
    pkgs = import nixpkgs {
      inherit system overlays;
    };
  in
{
devShells.default = pkgs.mkShell {
  buildInputs = (with pkgs;[
    # Basic
    openssl pkg-config fd zlib gnumake
    # Development tools
    ripgrep fzf zellij
    # Rust
    rustup
    cargo-binutils
    rust-toolchain
    # Test
    apacheHttpd
  ]) ++ [
  # Overlays part
    pkgs.qemu
  ];
  
  shellHook = ''
    alias find=fd
    export SHELL=zsh

    # Change the mirror of rust
    export RUSTUP_DIST_SERVER=https://mirrors.ustc.edu.cn/rust-static
    export RUSTUP_UPDATE_ROOT=https://mirrors.ustc.edu.cn/rust-static/rustup

    unset OBJCOPY # Avoiding Overlay
    export LIBCLANG_PATH="${pkgs.llvmPackages.libclang.lib}/lib" # nixpkgs@52447
    export LD_LIBRARY_PATH="${pkgs.zlib}/lib:$LD_LIBRARY_PATH" # nixpkgs@92946
  
    export PATH=$PATH:${pkgs.aarch64-linux-musl-cross}/bin:
    ${pkgs.riscv64-linux-musl-cross}/bin:${pkgs.x86_64-linux-musl-cross}/bin
  '';
};
};
  );
}      
\end{lstlisting}

\begin{lstlisting}[language=C
        , caption={初始化 RX Rings}
        , label={code::InitRxRings}
        , numbers = left
        , breaklines=true
        , captionpos=b
        , breakatwhitespace=true]
static int bcmgenet_init_rx_ring(struct bcmgenet_priv *priv, unsigned int index, unsigned int size, unsigned int start_ptr, unsigned int end_ptr);
for (i = 0; i < priv->hw_params->rx_queues; i++) { // 初始化函数
ret = bcmgenet_init_rx_ring(priv, i, priv->hw_params->rx_bds_per_q,
              i * priv->hw_params->rx_bds_per_q, (i + 1) *
              priv->hw_params->rx_bds_per_q);
if (ret) return ret;
ring_cfg |= (1 << i);
dma_ctrl |= (1 << (i + DMA_RING_BUF_EN_SHIFT));
\end{lstlisting}

\begin{lstlisting}[language=Rust
  , caption={tock register 包装}
  , label={code::tock_register}
  ]
  register_structs! {
    Channel {
        (0x00 => CS: ReadWrite<u32, CS::Register>),
        (0x04 => CONBLK: ReadWrite<u32, CONBLK::Register>),
        (0x08 => TI: ReadWrite<u32, TI::Register>),
        (0x0c => S_AD: ReadWrite<u32, S_AD::Register>),
        (0x10 => D_AD: ReadWrite<u32, D_AD::Register>),
        ...
      },
  } ...
  register_bitfields! { u32,
  CS [ // Control and Status registers
    RESET OFFSET(31) NUMBITS(1),
    ABORT OFFSET(30) NUMBITS(1),
    DISDEBUG OFFSET(29) NUMBITS(1),
    WAIT_FOR_OUTS_TANDING_WRITES OFFSET(28) NUMBITS(1),
    PANIC_PRIORITY OFFSET(20) NUMBITS(3),
    PRIORITY OFFSET(16) NUMBITS(4),
    ...
  ], CS_DMA4 [ ... ] ...}
\end{lstlisting}
  % 附件A,如果没有可以将此部分内容做成注释或删除即可 
% 
%\nonumsection{后记}  % 如果不需要,请将此部分内容做成注释或删除即可
%后记一般用于就某个问题提出引人深思的看法,让读者能够进行更深层次的思考。

\end{document}
